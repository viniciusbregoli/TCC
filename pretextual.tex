% ---
% RESUMOS
% ---

% Resumo em português
\setlength{\absparsep}{18pt}
\begin{resumo}

	A tomada de decisão no curto prazo em ambientes complexos, como
	centros cirúrgicos, exige ferramentas capazes de integrar dados
	históricos e informações em tempo real. Nesse contexto, a mineração
	de processos (Process Mining -- PM) e a simulação computacional
	emergem como tecnologias para apoiar gestores na alocação eficiente
	de recursos, detecção de desvios e previsão de cenários. Este
	trabalho propõe o desenvolvimento de um gerador de modelos de
	simulação baseado em PM, com foco em reduzir o esforço humano e o
	tempo necessário para a construção de modelos. O gerador utiliza logs
	de eventos no formato XES como fonte de dados para criar
	automaticamente modelos de simulação, permitindo avaliar alternativas
	de curto prazo e apoiar a tomada de decisão operacional. Como
	resultado, foi desenvolvido um protótipo funcional que integra cinco
	funcionalidades principais: análise automática de logs, mineração de
	processos utilizando o algoritmo Inductive Miner, simulação de
	eventos com geração de logs sintéticos, validação comparativa entre
	logs originais e simulados, e cálculo de indicadores ORE (Operating
	Room Effectiveness). O protótipo foi implementado como uma interface
	web interativa utilizando Streamlit, e validado com dados reais de
	agendamentos cirúrgicos de um centro cirúrgico. A ferramenta permite
	aos gestores extrair automaticamente modelos de processo (redes de
	Petri e árvores de processo), gerar cenários simulados para análise
	de capacidade e desempenho, e calcular métricas de efetividade
	operacional, contribuindo para maior eficiência, redução de custos e
	melhor qualidade no atendimento. \vspace{\onelineskip}

	\noindent
	\textbf{Palavras-chave}: Mineração
	de Processos, Simulação Computacional, Tomada de Decisão, Otimização,
	Agendamento.
\end{resumo}

% Resumo em inglês
\begin{resumo}[Abstract]
	\begin{otherlanguage*}{english}
		Short-term decision-making in complex environments, such as surgical centers, requires tools capable of integrating historical data and real-time information. In this context, Process Mining (PM) and computer simulation emerge as key technologies to support managers in efficient resource allocation, deviation detection, and scenario prediction. This work proposes the development of a simulation model generator based on PM, focusing on reducing human effort and the time required to build models. The generator uses event logs in XES format as a data source to automatically create simulation models, allowing the evaluation of short-term alternatives and supporting operational decision-making. As a result, a functional prototype was developed integrating five main functionalities: automatic log analysis, process mining using the Inductive Miner algorithm, event simulation with synthetic log generation, comparative validation between original and simulated logs, and calculation of ORE (Operating Room Effectiveness) indicators. The prototype was implemented as an interactive web interface using Streamlit, and validated with real surgical scheduling data from a surgical center. The tool enables managers to automatically extract process models (Petri nets and process trees), generate simulated scenarios for capacity and performance analysis, and calculate operational effectiveness metrics, contributing to greater efficiency, cost reduction, and improved service quality.
		\vspace{\onelineskip}

		\noindent
		\textbf{Keywords}: Process Mining, Computer Simulation, Decision-Making, Optimization, Scheduling.
	\end{otherlanguage*}
\end{resumo}

% ---
% Lista de ilustrações
% ---
\pdfbookmark[0]{\listfigurename}{lof}
\listoffigures*
\cleardoublepage

% ---
% Lista de tabelas
% ---
\pdfbookmark[0]{\listtablename}{lot}
\listoftables*
\cleardoublepage

% ---
% Sumário
% ---
\pdfbookmark[0]{\contentsname}{toc}
\tableofcontents*
\cleardoublepage

% ----------------------------------------------------------
% ELEMENTOS TEXTUAIS
% ----------------------------------------------------------
\textual
