\documentclass[
    12pt,               % Tamanho da fonte
    openright,          % Capítulos começam em página ímpar
    oneside,            % Impressão em um lado
    a4paper,            % Tamanho do papel
    brazil              % Idioma principal
]{abntex2}

% ---
% Pacotes básicos 
% ---
\usepackage{times}             % Usa fonte Times New Roman
\usepackage[T1]{fontenc}        % Seleção de códigos de fonte
\usepackage[utf8]{inputenc}     % Codificação do documento
\usepackage{indentfirst}        % Indenta o primeiro parágrafo
\usepackage{color}              % Controle das cores
\usepackage{graphicx}           % Inclusão de gráficos
\usepackage{microtype}          % Para melhorias de justificação
\emergencystretch=2em
\usepackage{lastpage}           % Para referência à última página
\usepackage{listings}           % Para blocos de código
\lstset{
    breaklines=true,
    breakatwhitespace=true,
    basicstyle=\small\ttfamily,
    columns=flexible,
    inputencoding=utf8,
    extendedchars=true
}
\renewcommand{\lstlistingname}{Código}
\usepackage{amsmath}           % Para ambientes matemáticos
\usepackage{algorithm}         % Para ambiente de algoritmos
\usepackage{algpseudocode}     % Para pseudocódigo

% ---
% Pacotes de citações
% ---
\usepackage[brazilian,hyperpageref]{backref}     % Páginas com as citações
\usepackage[alf]{abntex2cite}   % Citações padrão ABNT

% ---
% Configurações do pacote backref
% ---
\renewcommand{\backrefpagesname}{Citado na(s) página(s):~}
\renewcommand{\backref}{}
\renewcommand*{\backrefalt}[4]{
    \ifcase #1 %
        Nenhuma citação no texto.%
    \or
        Citado na página #2.%
    \else
        Citado #1 vezes nas páginas #2.%
    \fi}%

% ---
% Informações do documento
% ---
\titulo{DESENVOLVIMENTO DE GERADOR DE MODELOS DE SIMULAÇÃO PARA TOMADA DE DECISÃO NO CURTO PRAZO USANDO PROCESS MINING}
\autor{VINÍCIUS LEOBET BREGOLI}
\local{Curitiba}
\data{2025}
\orientador{Prof. Dr. Edson Emílio Scalabrin}
\instituicao{%
  PONTIFÍCIA UNIVERSIDADE CATÓLICA DO PARANÁ -- PUCPR
  \par
  ESCOLA POLITÉCNICA
  \par
  CURSO DE ENGENHARIA DE COMPUTAÇÃO}
\tipotrabalho{Trabalho de Conclusão de Curso}
\preambulo{Trabalho de Conclusão de Curso apresentado ao Curso de Engenharia de Computação da Pontifícia Universidade Católica do Paraná como requisito parcial para obtenção do grau de Bacharel em Engenharia de Computação.}

% ---
% Configurações de aparência do PDF
% ---
\definecolor{blue}{RGB}{41,5,195}

\makeatletter
\hypersetup{
    pdftitle={\@title},
    pdfauthor={\@author},
    pdfsubject={\imprimirpreambulo},
    pdfcreator={LaTeX with abnTeX2},
    pdfkeywords={abnt}{latex}{tcc}{engenharia de computação},
    colorlinks=true,
    linkcolor=blue,
    citecolor=blue,
    filecolor=magenta,
    urlcolor=blue,
    bookmarksdepth=4,
    pdfstartview=FitH,
    unicode=true
}
\makeatother

% Corrigir warnings do hyperref para títulos com acentos
\pdfstringdefDisableCommands{%
  \def\\{}%
  \def\textbf#1{<#1>}%
  \def\textit#1{<#1>}%
  \def\uppercase#1{#1}%
}

% ---
% Configurações de formatação ABNT
% ---

% Margens: 3cm superior e esquerda, 2cm inferior e direita
\usepackage[
    top=3cm,
    left=3cm,
    right=2cm,
    bottom=2cm
]{geometry}

% Espaçamento 1,5 no texto
\usepackage{setspace}

% Configurações de parágrafo
\setlength{\parindent}{1.3cm}
\setlength{\parskip}{0.2cm}

% Justificação do texto (já é padrão no LaTeX, mas garantindo)
\usepackage{ragged2e}
\justifying

% Garantir que títulos também usem Times New Roman (ABNT exige mesma fonte)
\renewcommand{\ABNTEXchapterfont}{\rmfamily\bfseries}
\renewcommand{\ABNTEXsectionfont}{\rmfamily\bfseries}
\renewcommand{\ABNTEXsubsectionfont}{\rmfamily\bfseries}
\renewcommand{\ABNTEXsubsubsectionfont}{\rmfamily\bfseries}

% Configuração para citações longas (espaçamento simples)
\renewenvironment{citacao}{%
    \small
    \begin{singlespace}
    \begin{list}{}{%
        \setlength{\leftmargin}{4cm}%
        \setlength{\parsep}{0pt}%
        \setlength{\parskip}{0pt}%
        \setlength{\itemsep}{0pt}%
    }
    \item\relax
}{%
    \end{list}
    \end{singlespace}
}

% ---
% Compila o índice
% ---
\makeindex

% ----
% Início do documento
% ----
\begin{document}

\selectlanguage{brazil}
\frenchspacing

% ----------------------------------------------------------
% ELEMENTOS PRÉ-TEXTUAIS
% ----------------------------------------------------------

% ---
% Capa
% ---
\imprimircapa

% ---
% Folha de rosto
% ---
\imprimirfolhaderosto*

% ---
% Elementos pré-textuais
% ---
% ---
% RESUMOS
% ---

% Resumo em português
\setlength{\absparsep}{18pt}
\begin{resumo}

	A tomada de decisão no curto prazo em ambientes complexos, como
	centros cirúrgicos, exige ferramentas capazes de integrar dados
	históricos e informações em tempo real. Nesse contexto, a mineração
	de processos (Process Mining -- PM) e a simulação computacional
	emergem como tecnologias para apoiar gestores na alocação eficiente
	de recursos, detecção de desvios e previsão de cenários. Este
	trabalho propõe o desenvolvimento de um gerador de modelos de
	simulação baseado em PM, com foco em reduzir o esforço humano e o
	tempo necessário para a construção de modelos. O gerador utiliza logs
	de eventos no formato XES como fonte de dados para criar
	automaticamente modelos de simulação, permitindo avaliar alternativas
	de curto prazo e apoiar a tomada de decisão operacional. Como
	resultado, foi desenvolvido um protótipo funcional que integra cinco
	funcionalidades principais: análise automática de logs, mineração de
	processos utilizando o algoritmo Inductive Miner, simulação de
	eventos com geração de logs sintéticos, validação comparativa entre
	logs originais e simulados, e cálculo de indicadores ORE (Operating
	Room Effectiveness). O protótipo foi implementado como uma interface
	web interativa utilizando Streamlit, e validado com dados reais de
	agendamentos cirúrgicos de um centro cirúrgico. A ferramenta permite
	aos gestores extrair automaticamente modelos de processo (redes de
	Petri e árvores de processo), gerar cenários simulados para análise
	de capacidade e desempenho, e calcular métricas de efetividade
	operacional, contribuindo para maior eficiência, redução de custos e
	melhor qualidade no atendimento. \vspace{\onelineskip}

	\noindent
	\textbf{Palavras-chave}: Mineração
	de Processos, Simulação Computacional, Tomada de Decisão, Otimização,
	Agendamento.
\end{resumo}

% Resumo em inglês
\begin{resumo}[Abstract]
	\begin{otherlanguage*}{english}
		Short-term decision-making in complex environments, such as surgical centers, requires tools capable of integrating historical data and real-time information. In this context, Process Mining (PM) and computer simulation emerge as key technologies to support managers in efficient resource allocation, deviation detection, and scenario prediction. This work proposes the development of a simulation model generator based on PM, focusing on reducing human effort and the time required to build models. The generator uses event logs in XES format as a data source to automatically create simulation models, allowing the evaluation of short-term alternatives and supporting operational decision-making. As a result, a functional prototype was developed integrating five main functionalities: automatic log analysis, process mining using the Inductive Miner algorithm, event simulation with synthetic log generation, comparative validation between original and simulated logs, and calculation of ORE (Operating Room Effectiveness) indicators. The prototype was implemented as an interactive web interface using Streamlit, and validated with real surgical scheduling data from a surgical center. The tool enables managers to automatically extract process models (Petri nets and process trees), generate simulated scenarios for capacity and performance analysis, and calculate operational effectiveness metrics, contributing to greater efficiency, cost reduction, and improved service quality.
		\vspace{\onelineskip}

		\noindent
		\textbf{Keywords}: Process Mining, Computer Simulation, Decision-Making, Optimization, Scheduling.
	\end{otherlanguage*}
\end{resumo}

% ---
% Lista de ilustrações
% ---
\pdfbookmark[0]{\listfigurename}{lof}
\listoffigures*
\cleardoublepage

% ---
% Lista de tabelas
% ---
\pdfbookmark[0]{\listtablename}{lot}
\listoftables*
\cleardoublepage

% ---
% Sumário
% ---
\pdfbookmark[0]{\contentsname}{toc}
\tableofcontents*
\cleardoublepage

% ----------------------------------------------------------
% ELEMENTOS TEXTUAIS
% ----------------------------------------------------------
\textual


% ----------------------------------------------------------
% ELEMENTOS TEXTUAIS
% ----------------------------------------------------------
\chapter{Introdução}

Organizações modernas enfrentam o desafio de transformar o crescente volume de dados operacionais em decisões rápidas e baseadas em evidências. Enquanto sistemas informatizados geram logs detalhados de execução de processos, a análise desses dados permanece majoritariamente manual e dependente de expertise humana. A \textit{mineração de processos} (PM) permite extrair modelos descritivos de processos reais, mas oferece primariamente uma visão retrospectiva. A \textit{simulação de eventos discretos} (DES), por outro lado, viabiliza a análise preditiva de cenários, porém sua construção manual é intensiva e inviável para decisões de curto prazo.

O presente trabalho desenvolveu um \textbf{gerador automatizado de modelos de simulação baseado em mineração de processos}, capaz de transformar logs de eventos estruturados em modelos DES parametrizados sem intervenção manual. O protótipo foi validado com dados reais do setor hospitalar e projetado para ser generalizável a diferentes domínios organizacionais que disponham de registros no padrão XES. Essa integração preenche lacuna identificada na literatura: a necessidade de automação end-to-end entre extração de conhecimento (PM) e predição operacional (DES).

\section{Contexto e Problema}

A crescente complexidade dos ambientes organizacionais exige capacidade contínua de adaptação e resposta rápida. Setores como saúde, manufatura e logística compartilham desafios recorrentes: alocação eficiente de recursos, detecção de gargalos e melhoria contínua de desempenho. Em todos esses contextos, decisões de curto prazo — aquelas que precisam ser tomadas em intervalos de horas ou dias — exercem influência direta sobre a produtividade e eficiência operacional.

Com a intensificação da digitalização, grandes volumes de dados passaram a ser gerados em tempo real. Contudo, a transformação desses dados em conhecimento útil ainda depende, em grande medida, da experiência humana e de análises manuais, sujeitas a vieses cognitivos e limitações de tempo de resposta.

A \textit{mineração de processos} (PM) consolida-se como abordagem para extração de conhecimento estruturado a partir de logs de eventos, com aplicações demonstradas em domínios complexos. Entretanto, PM oferece predominantemente uma visão descritiva e diagnóstica. A simulação de eventos discretos, por outro lado, permite avaliar cenários alternativos, mas sua construção manual é intensiva e inviável para decisões de curto prazo que exigem reação rápida a eventos inesperados.

O desafio central, portanto, é: \textbf{como automatizar a geração de modelos de simulação baseados em dados reais, de modo a apoiar decisões operacionais de curto prazo}? Trabalhos como PM4SOS \cite{ferronato2022} demonstraram potencial dessa integração em contextos específicos (hospitalar), mas ainda com dependência de intervenções manuais e limitada generalização. A lacuna identificada reside na necessidade de uma solução automatizada, end-to-end e generalizável, que reduza o esforço de modelagem e amplie a aplicabilidade da técnica ao suporte à decisão operacional.

\section{Delimitação do Escopo}

\textbf{Escopo técnico:} Este trabalho aborda processos discretos com eventos estruturados no padrão XES. A solução integra descoberta de processos (Inductive Miner via PM4PY), extração estatística de distribuições de tempos, identificação automática de recursos, e geração de modelos parametrizados em SimPy.

\textbf{Escopo temporal:} O foco é apoiar decisões operacionais de curto prazo, definidas como aquelas que necessitam resposta em horizonte de horas a dias (não inclui planejamento estratégico de longo prazo ou otimização em tempo real).

\textbf{Escopo de aplicação:} Aplicável a processos organizacionais que possuam logs de eventos estruturados conforme padrão XES, incluindo contextos hospitalares, administrativos, logísticos e de manufatura.

\textbf{Limitações explícitas:} O trabalho não aborda: (i) processos contínuos ou híbridos (contínuo-discreto), (ii) tratamento de sensores de tempo real não estruturados, (iii) otimização multicritério automática (focos apenas em simulação), (iv) modelos com sincronização complexa entre múltiplos recursos, (v) tratamento de incerteza estocástica além das distribuições estatísticas estimadas.

\section{Motivação}

Organizações modernas enfrentam uma urgência operacional: decisões precisam ser tomadas em horas ou dias, não em semanas. A maioria dos dados operacionais coletados continua subutilizada, pois análises descritivas retrospectivas não oferecem suporte efetivo a decisões ágeis. O framework PM4SOS \cite{ferronato2022} demonstrou que a integração entre mineração de processos e simulação é viável em contextos hospitalares, porém permanece restrita e com etapas manuais.

A motivação deste trabalho é clara: criar uma solução que (i) elimine intervenções manuais na geração de modelos, (ii) reduza o tempo de análise de semanas para minutos, (iii) seja aplicável a diferentes domínios sem reprogramação, e (iv) democratize o acesso a ferramentas de simulação para analistas e gestores sem expertise em modelagem formal. Acredita-se que automatização completa do pipeline PM-DES pode transformar simulação de instrumento ocasional de especialistas em ferramenta cotidiana de apoio à decisão operacional.

\section{Soluções Similares}

Diversas pesquisas têm buscado integrar \textit{Process Mining} (PM)
e Simulação de Eventos Discretos como forma de
aprimorar a compreensão e a predição do comportamento dos processos
reais. Entretanto, a maioria das soluções existentes mantém um alto
grau de dependência de intervenção manual, especialmente nas etapas
de modelagem, parametrização e calibração.

Entre as iniciativas mais influentes, destaca-se a metodologia
proposta por \cite{maruster2009redesigning}, que combina mineração de
processos e simulação para o redesenho organizacional. O método parte
de logs reais para gerar modelos \textit{As-Is}, simulá-los e
compará-los com versões otimizadas \textit{To-Be}, permitindo estimar
ganhos de desempenho. Apesar de pioneiro, o processo de conversão dos
modelos minerados em modelos simuláveis requer ajustes manuais e
conhecimento técnico em modelagem formal (como redes de Petri
coloridas).

No contexto hospitalar, o framework PM4SOS, desenvolvido por
\cite{ferronato2022}, integra mineração de processos, simulação e
otimização multicritério para o agendamento cirúrgico. Essa abordagem
automatiza parcialmente a geração de modelos e utiliza indicadores de
eficiência para suportar decisões em tempo reduzido. Contudo, sua
aplicação ainda é restrita ao domínio da saúde, carecendo de
generalização para outros tipos de processos.

Na área industrial, trabalhos como \cite{wuennenberg2023internal}
propõem pipelines que unem simulação e mineração de processos em
sistemas logísticos internos. Esses modelos exploram o uso de
simulação para geração de dados sintéticos, que são posteriormente
minerados para verificação de conformidade e detecção de gargalos. Em
paralelo, pesquisas em mineração subterrânea
\cite{brzychczy2024pm4lmp} e simulação modular
\cite{meng2024enhancing} também avançam na integração entre dados de
sensores, abstração de eventos e análise preditiva, embora com foco
em contextos físicos específicos.

Em síntese, as soluções atuais demonstram o potencial da integração
entre PM e DES, mas permanecem limitadas quanto à automação de ponta
a ponta e à adaptabilidade entre diferentes domínios. O presente
trabalho propõe evoluir essas abordagens por meio de um gerador de
modelos de simulação automatizado e generalizável, reduzindo o
esforço técnico necessário e ampliando o alcance da análise preditiva
em processos de decisão operacional de curto prazo.

\section{Objetivos}

\subsection{Objetivo Geral}

Desenvolver um \textbf{gerador de modelos de simulação baseado em
	mineração de processos} para apoiar a \textbf{tomada de decisão no
	curto prazo}, capaz de criar automaticamente modelos de simulação a
partir de logs de eventos, reduzindo o esforço humano e o tempo
necessário para a modelagem de sistemas complexos.

\subsection{Objetivos Específicos}

Para alcançar o objetivo geral, este trabalho busca:

\begin{itemize}
	\item \textbf{Analisar} métodos de integração PM-DES, documentando grau de automação, precisão de modelos e tempo de execução de pelo menos 3 abordagens existentes;
	\item \textbf{Projetar} arquitetura modular que suporte: (a) leitura automática de logs XES, (b) descoberta de processos via Inductive Miner, (c) extração estatística de distribuições de tempos (normal, lognormal, exponencial), (d) identificação automática de recursos, (e) geração de código SimPy parametrizado;
	\item \textbf{Validar} em estudo de caso hospitalar real: demonstrar correspondência entre comportamento simulado e dados históricos com fitness score $\geq 0,80$ e precisão $\geq 0,75$;
	\item \textbf{Avaliar} potencial de generalização através de testes em pelo menos 2 domínios adicionais (administrativo/logístico) e documentar limitações técnicas identificadas.
\end{itemize}

\section{Contribuições Principais}

Este trabalho apresenta as seguintes contribuições:

\begin{itemize}
	\item \textbf{Arquitetura automatizada end-to-end:} Pipeline integrado que transforma logs XES em modelos DES parametrizados sem intervenção manual, eliminando lacuna de 2-3 semanas típica de modelagem convencional para execução em minutos.
	\item \textbf{Protocolo de parametrização estatística:} Método sistemático para extração automática de distribuições de tempos, identificação de recursos e padrões de decisão diretamente de event logs, garantindo reprodutibilidade e consistência.
	\item \textbf{Framework generalizável cross-domain:} Solução testada em contexto hospitalar mas aplicável a manufatura, logística, administração e serviços, desde que dados estejam estruturados em padrão XES.
	\item \textbf{Validação com dados reais:} Protótipo validado com logs históricos de processos cirúrgicos, demonstrando fitness e precisão compatíveis com literatura, comprovando viabilidade operacional.
	\item \textbf{Redução de esforço humano:} Democratização do acesso a simulação como ferramenta de apoio à decisão, ampliando capacidade analítica de organizações sem expertise em modelagem formal.
\end{itemize}

\section{Justificativa}

A literatura demonstra potencial da integração PM-DES \cite{maruster2009redesigning, wuennenberg2023internal, ferronato2022}, porém com dependência de etapas manuais em modelagem, parametrização e calibração. Este trabalho avança teoricamente ao propor método completamente automatizado, com protocolo replicável e reprodutível, ampliando o corpo de conhecimento sobre como transformar dados históricos em modelos preditivos sem perda de precisão. Contribui para reconciliar a natureza descritiva de PM com a capacidade preditiva de DES em contexto operacional.

O protótipo representa avanço em relação a PM4SOS \cite{ferronato2022} e metodologias similares. Enquanto trabalhos precedentes \cite{maruster2009redesigning, wuennenberg2023internal} requerem 2-3 semanas de modelagem manual especializada, esta solução automatiza o pipeline completo em minutos. Diferencia-se por: (i) eliminação de intervenção manual, (ii) generalizabilidade cross-domain (não restrita a saúde), (iii) parametrização estatística sistemática com distribuições ajustadas aos dados (normal, lognormal, exponencial), (iv) integração em ferramenta única e acessível.

Organizações modernas enfrentam pressão por decisões ágeis. Dados operacionais abundam, mas análises demoram semanas. A solução proposta reduz este ciclo para minutos, viabilizando uso operacional cotidiano. Validação com dados reais de processos cirúrgicos comprova viabilidade e aplicabilidade em contextos operacionais complexos \cite{ferronato2022}. Potencial de aplicação: saúde, manufatura, logística, administração — qualquer domínio com logs estruturados XES. Amplia a capacidade analítica organizacional sem exigir expertise em modelagem formal, democratizando acesso a ferramentas de simulação.
% ----------------------------------------------------------
% Referencial Teórico
% ----------------------------------------------------------

\chapter{Referencial Teórico}

Este capítulo apresenta os fundamentos conceituais e técnicos que
sustentam o desenvolvimento de um gerador de modelos de simulação
orientado por mineração de processos. São
abordados desde os conceitos fundamentais de mineração de processos
até a integração com simulação de eventos discretos e indicadores de
desempenho operacional aplicados a ambientes hospitalares.

\section{Mineração de Processos}

A mineração de processos  é uma disciplina
que une conceitos de mineração de dados e de
gerenciamento de processos de negócio com o objetivo de extrair conhecimento útil a
partir de registros de eventos provenientes de
sistemas de informação. Segundo van der Aalst
\cite{aalst2016process}, a mineração de processos tem como propósito
descobrir, monitorar e aprimorar processos reais com base nos dados
efetivamente registrados, constituindo uma ponte entre a análise
orientada a dados e a modelagem formal de processos.

Com a crescente digitalização das operações empresariais e o avanço
de sistemas como ERPs, CRMs e sistemas de controle de manufatura,
passou a ser possível registrar detalhadamente cada etapa executada
em um processo. Esses registros, conhecidos como \textit{event logs},
formam a base para a aplicação de técnicas de mineração de processos.
Cada evento registrado representa a execução de uma atividade
pertencente a um caso e contém atributos como o nome
da atividade, o identificador do caso, o responsável
(\textit{resource}) e o carimbo de tempo (\textit{timestamp}). A
estrutura desses logs permite a reconstituição da sequência de
atividades e o estudo do comportamento do processo real
\cite{maruster2009redesigning}.

A mineração de processos é composta por três grandes categorias de
técnicas \cite{aalst2016process}:
\begin{enumerate}
	\item \textbf{Descoberta de processos (\textit{Process Discovery})}: tem como objetivo gerar automaticamente um modelo de processo a partir de um log de eventos, sem conhecimento prévio do fluxo. O modelo resultante pode ser representado em diferentes notações, como Redes de Petri, BPMN (\textit{Business Process Model and Notation}) ou Árvores de Processo (\textit{Process Trees}).
	\item \textbf{Verificação de conformidade (\textit{Conformance Checking})}: consiste em comparar um modelo de processo pré-existente com um log de eventos real, avaliando a aderência entre o comportamento observado e o comportamento esperado. Essa comparação fornece métricas como \textit{fitness} e \textit{precision}.
	\item \textbf{Aprimoramento de modelos (\textit{Enhancement})}: busca enriquecer modelos existentes com informações adicionais provenientes dos logs, como tempos médios de execução, gargalos, desvios e uso de recursos, permitindo a análise de desempenho e a detecção de oportunidades de otimização.
\end{enumerate}

O ciclo de mineração de processos, descrito no \textit{Process Mining
	Manifesto} da IEEE Task Force on Process Mining
\cite{aalst2016process}, enfatiza que a aplicação prática da técnica
depende da disponibilidade e da qualidade dos logs de eventos. Logs
incompletos, inconsistentes ou mal formatados comprometem a acurácia
dos modelos descobertos. Nesse sentido, Kherbouche et al.
\cite{betterlogs2020} destacam a importância da avaliação da
qualidade dos logs, propondo métricas de \textit{completude},
\textit{consistência} e \textit{complexidade} antes da aplicação das
técnicas de mineração.

Em termos de arquitetura, um sistema de mineração de processos segue
um fluxo básico: coleta de dados, pré-processamento, mineração
propriamente dita e análise dos resultados. Durante a coleta, os logs
podem ser extraídos de sistemas como SAP, Oracle, ou de bancos de
dados customizados. O pré-processamento envolve a limpeza e
padronização dos dados, incluindo a identificação correta de casos e
atividades. Em seguida, algoritmos como \textit{Alpha Miner},
\textit{Heuristic Miner} e \textit{Inductive Miner} são aplicados
para a descoberta do modelo de processo
\cite{leemans2013discovering}.

Os resultados podem ser apresentados em notações formais, como Redes
de Petri, ou em linguagens mais visuais, como BPMN. O uso de
ferramentas especializadas, como o \textit{ProM Framework} e a
biblioteca Python PM4Py, facilita essa análise e integração com
outros métodos, como a simulação de eventos discretos
\cite{ferronato2021pm2sim}.

O padrão \textit{eXtensible Event Stream} (XES), definido pela IEEE
\cite{ieeexes2010}, estabelece a estrutura de logs de eventos para
garantir interoperabilidade entre ferramentas e consistência na troca
de dados. Cada log em XES é composto por uma coleção de
\textit{traces} (casos), e cada \textit{trace} contém uma sequência
ordenada de \textit{events}. Essa padronização é essencial para o
sucesso de frameworks modernos de mineração de processos e simulação
integrada.

Portanto, a mineração de processos se consolida como uma abordagem
essencial para compreender, auditar e melhorar processos
organizacionais em ambientes baseados em dados. Quando combinada com
simulação e análise estatística, ela se torna uma ferramenta poderosa
para suporte à decisão, otimização operacional e melhoria contínua de
processos complexos, como os hospitalares e logísticos.

\section{Inductive Miner}

Entre os diversos algoritmos de descoberta de processos disponíveis,
o \textit{Inductive Miner} (IM), proposto por Leemans, Fahland e van
der Aalst \cite{leemans2013discovering}, é um dos mais relevantes e
amplamente utilizados na literatura e em ferramentas modernas de
mineração, como o \textit{ProM} e o \texttt{PM4Py}.

Diferentemente de abordagens anteriores, como o \textit{Alpha Miner}
e o \textit{Heuristic Miner}, o Inductive Miner foi projetado para
garantir propriedades formais no modelo resultante, como a
\textit{soundness} (correção comportamental) e a estrutura
hierárquica em blocos. Essas propriedades
asseguram que o modelo possa ser executado sem estados mortos ou
impasses, além de facilitar sua conversão em linguagens formais como
Redes de Petri e Árvores de Processo.

O princípio fundamental do algoritmo baseia-se na decomposição
recursiva do log de eventos. O IM analisa as relações de precedência
e causalidade entre atividades e divide o log em sublogs coerentes,
representando fragmentos independentes do processo. Cada sublog é
então minerado de forma isolada, e os resultados são combinados em
uma estrutura hierárquica que reflete a composição lógica das
atividades.

Uma das vantagens práticas do Inductive Miner é sua compatibilidade
direta com representações de simulação. Ao produzir modelos
formalmente corretos e livres de inconsistências estruturais, o IM
facilita a transformação automática em modelos de simulação baseados
em Redes de Petri, como destacado por Ferronato
\cite{ferronato2021pm2sim}. Essa característica é essencial para o
desenvolvimento de sistemas como o \textit{PM2Sim}, que automatiza a
criação de modelos de simulação de eventos discretos.

\section{Redes de Petri}

As Redes de Petri constituem um formalismo matemático e gráfico
amplamente utilizado para modelar, analisar e simular sistemas de
eventos discretos. Propostas originalmente por Carl Adam Petri na
década de 1960 e formalizadas por Peterson \cite{peterson1981petri},
essas redes oferecem uma base rigorosa para a representação de
processos dinâmicos caracterizados por concorrência, sincronização,
conflito e causalidade — propriedades típicas de sistemas produtivos,
logísticos e hospitalares.

Uma Rede de Petri é definida formalmente como uma tupla $N = (P, T,
	F)$, onde:
\begin{itemize}
	\item $P$ representa o conjunto de lugares (\textit{places});
	\item $T$ representa o conjunto de transições (\textit{transitions});
	\item $F \subseteq (P \times T) \cup (T \times P)$ é o conjunto de arcos que conectam lugares e transições.
\end{itemize}

Os lugares simbolizam condições ou estados do sistema, enquanto as
transições representam eventos ou atividades que modificam esses
estados. A dinâmica do sistema é descrita por meio da movimentação de
fichas (\textit{tokens}) entre os lugares. O conjunto de tokens em um
dado instante define a marcação atual da rede (\textit{marking}),
representando o estado do processo. Quando todas as condições de
disparo de uma transição são satisfeitas, ela se torna habilitada e
pode ser executada, consumindo e produzindo tokens conforme o fluxo
definido em $F$. Essa semântica de disparo possibilita a modelagem de
sistemas paralelos e assíncronos de forma intuitiva e precisa.

Segundo Peterson \cite{peterson1981petri}, uma das principais
vantagens das Redes de Petri é a possibilidade de realizar análises
formais de propriedades do sistema modelado, como:
\begin{itemize}
	\item \textbf{Alcançabilidade} (\textit{Reachability}): determina quais estados podem ser atingidos a partir da marcação inicial.
	\item \textbf{Viveza} (\textit{Liveness}): garante que nenhuma transição se torne permanentemente inativa, evitando estados mortos.
	\item \textbf{Conservação} (\textit{Boundedness}): assegura que o número de tokens em cada lugar permanece finito, prevenindo explosões de estados.
	\item \textbf{Deadlock-freedom}: certifica que o sistema não entra em impasse completo.
\end{itemize}

Essas propriedades são fundamentais para a verificação de correção
comportamental (\textit{soundness}) em modelos de processos
descobertos via mineração, assegurando que o modelo é executável e
que todo caso pode ser concluído corretamente
\cite{aalst2016process}.

No contexto da mineração de processos, as Redes de Petri são
amplamente utilizadas como formalismo intermediário para representar
os modelos extraídos de logs de eventos. O \textit{Inductive Miner},
por exemplo, gera diretamente uma Rede de Petri \textit{sound} e
\textit{block-structured} \cite{leemans2013discovering}, permitindo
tanto a verificação de conformidade quanto a execução simulada do
processo. Essa característica é essencial para a integração entre
mineração e simulação de eventos discretos, pois possibilita a
tradução direta de modelos minerados em estruturas simuláveis.

\begin{figure}[htb]
	\centering
	\includegraphics[width=0.8\textwidth]{figuras/fig1.png}
	\caption{Exemplo de Rede de Petri representando um processo de negócio. Fonte: Autor (2025)}
	\label{fig:rede_petri_exemplo}
\end{figure}

Na Figura~\ref{fig:rede_petri_exemplo}, os elementos da Rede de Petri
são representados da seguinte forma: os círculos representam
\textit{lugares} (places), que indicam estados ou condições do
processo; os retângulos representam \textit{transições}
(transitions), que correspondem a atividades ou eventos que podem
ocorrer; os círculos com pontos pretos internos representam
\textit{tokens} (marcas), que indicam a presença de uma condição ou o
estado atual do processo; e os círculos com contorno destacado e
tokens internos representam lugares marcados, indicando estados
ativos no momento atual da execução.

Ferronato \cite{ferronato2021pm2sim} destaca que as Redes de Petri
desempenham um papel central na integração entre mineração e
simulação. Em seu framework \textit{PM2Sim}, as redes extraídas a
partir de logs de eventos são automaticamente convertidas em modelos
de simulação de eventos discretos implementados em Python. Essa
conversão permite avaliar métricas como tempo de ciclo, gargalos e
utilização de recursos, transformando os modelos minerados em
instrumentos de suporte à decisão operacional.

Em síntese, as Redes de Petri constituem uma linguagem formal robusta
e expressiva para representar o comportamento de sistemas reais. Sua
adoção no contexto da mineração de processos garante não apenas a
fidelidade comportamental dos modelos gerados, mas também sua
viabilidade para análises de desempenho e simulação, consolidando-se
como elo fundamental entre a teoria de processos e a prática da
modelagem computacional.

\section{Simulação de Eventos Discretos}

A simulação de eventos discretos é uma técnica de modelagem computacional amplamente utilizada
para representar sistemas dinâmicos em que o estado do sistema muda
apenas em pontos discretos no tempo. Cada mudança é provocada por um
evento, que ocorre em um instante específico e representa uma
transição de estado. Essa abordagem é amplamente empregada em áreas
como manufatura, logística, saúde e serviços, onde os processos são
compostos por atividades sequenciais, paralelas e dependentes de
recursos.

A DES permite reproduzir o comportamento de sistemas complexos sem a
necessidade de interferir em seu ambiente real, possibilitando
análises de desempenho, previsão de gargalos e avaliação de cenários
alternativos. Diferentemente da simulação contínua, que modela
fenômenos em tempo contínuo por meio de equações diferenciais, a
simulação discreta descreve processos orientados por eventos, sendo,
portanto, ideal para a representação de fluxos de trabalho e
processos empresariais.

Em termos formais, um modelo de simulação de eventos discretos é
composto por três elementos básicos:
\begin{itemize}
	\item \textbf{Entidades}: representam os objetos que transitam pelo sistema (por exemplo, pacientes, ordens de serviço ou produtos).
	\item \textbf{Recursos}: correspondem aos elementos que executam as atividades (como profissionais, máquinas ou salas cirúrgicas).
	\item \textbf{Eventos}: são as ocorrências que alteram o estado do sistema, como o início ou o término de uma atividade.
\end{itemize}

Esses componentes são orquestrados por um \textit{motor de simulação}
(\textit{simulation engine}) que mantém um relógio lógico e agenda os
próximos eventos a serem processados. Cada evento executado pode
gerar novos eventos futuros, modificando o estado do sistema e
permitindo a evolução temporal do modelo \cite{liu2015integrating}.

A integração entre simulação e mineração de processos vem sendo
explorada nos últimos anos. Liu \cite{liu2015integrating} demonstrou
que os modelos extraídos via \textit{process mining} podem ser
automaticamente convertidos em modelos de simulação de eventos
discretos, reduzindo o esforço de modelagem e aumentando a precisão
analítica. Essa integração é especialmente relevante em contextos
onde a dinâmica do sistema se altera frequentemente, exigindo
atualizações rápidas de modelos e previsões.

Nesse contexto, Ferronato e Scalabrin \cite{ferronato2021pm2sim}
desenvolveram o framework \textit{PM2Sim}, que automatiza a criação
de modelos de simulação a partir de logs de eventos reais. O sistema
identifica atividades, durações, tempos de espera e recursos
envolvidos, transformando essas informações em um modelo DES
implementado em Python com a biblioteca \texttt{SimPy}. Essa
biblioteca fornece uma estrutura orientada a processos, baseada em
geradores, que permite modelar entidades como processos que interagem
em um ambiente temporal discreto. A \texttt{SimPy} oferece ainda
suporte à criação de filas, controle de recursos, eventos simultâneos
e coleta de estatísticas de desempenho, tornando-a ideal para
aplicações em mineração de processos e análise operacional.

Além disso, o framework \textit{PM2Sim} utiliza distribuições
estatísticas ajustadas por meio de técnicas de \textit{fitting}
(adequação), com base em bibliotecas como \texttt{NumPy} e
\texttt{SciPy}, permitindo representar o comportamento dos tempos de
execução das atividades. O resultado é um modelo de simulação que
reflete o comportamento observado nos logs, possibilitando a análise
de métricas como tempo de ciclo, utilização de recursos e
identificação de gargalos.

De acordo com Wuennenberg et al. \cite{wuennenberg2023internal}, a
combinação de mineração de processos e simulação discreta é
particularmente útil em ambientes industriais e logísticos, onde a
qualidade dos dados e a variabilidade operacional representam
desafios significativos. Nesse sentido, a DES serve como ferramenta
de validação e experimentação para modelos minerados, permitindo
testar hipóteses e prever o impacto de mudanças estruturais antes de
sua implementação no ambiente real.

\section{Análise Estatística}

A etapa de análise estatística é fundamental para a construção de
modelos de simulação realistas e coerentes com os processos
observados. Após a descoberta do modelo de processo e a extração das
informações de desempenho a partir dos logs de eventos, é necessário
realizar o ajuste estatístico dos parâmetros temporais, como durações
de atividades, tempos de espera e intervalos entre eventos. Esses
parâmetros são essenciais para que o modelo de simulação reflita
adequadamente a variabilidade e o comportamento estocástico do
sistema.

Segundo van der Aalst \cite{aalst2016process}, a mineração de
processos deve ser vista como uma disciplina orientada por dados
(\textit{data-driven}), e o sucesso de suas aplicações depende da
correta interpretação das distribuições de tempo, frequência e
desempenho que emergem dos registros de eventos. O uso de técnicas
estatísticas complementa a fase de descoberta, permitindo transformar
modelos descritivos em modelos quantitativos capazes de realizar
simulações e análises preditivas.

Ferronato e Scalabrin \cite{ferronato2021pm2sim} destacam que o uso
de distribuições estatísticas é um dos pilares do framework
\textit{PM2Sim}, que automatiza a criação de modelos de simulação a
partir de logs reais. No sistema proposto, os tempos de execução das
atividades e os intervalos entre eventos são ajustados a partir de
funções de probabilidade clássicas, como Normal, Log-Normal e
Exponencial. Essa parametrização é obtida por meio do ajuste de
distribuições (\textit{distribution fitting}), realizado com base nos
dados históricos de cada atividade registrada no log. O processo
envolve a estimativa dos parâmetros das distribuições e a verificação
de aderência dos dados observados, assegurando a representatividade
do modelo.

Para realizar esses ajustes, bibliotecas estatísticas como
\texttt{NumPy} e \texttt{SciPy} são utilizadas no ambiente Python,
permitindo estimar as distribuições de probabilidade e calcular
medidas como média, variância, desvio padrão e coeficiente de
variação. Além disso, testes de aderência, como o de
Kolmogorov–Smirnov, são empregados para validar se os dados amostrais
seguem adequadamente a distribuição teórica selecionada. Esse
procedimento garante que os tempos de simulação não apenas reproduzam
as médias históricas, mas também capturem a variabilidade natural do
processo \cite{liu2015integrating}.

O resultado da análise estatística é incorporado diretamente aos
parâmetros do modelo de simulação, alimentando o mecanismo de eventos
discretos com tempos amostrados de distribuições probabilísticas.
Essa abordagem permite representar o comportamento dinâmico e
imprevisível dos processos reais, algo essencial em ambientes
sujeitos a variações operacionais, como hospitais e sistemas
logísticos. Conforme observado por Leemans et al.
\cite{leemans2013discovering}, a precisão dos tempos e a correta
modelagem da frequência das atividades impactam diretamente a
capacidade do modelo minerado em reproduzir o fluxo real dos eventos.

Assim, a análise estatística atua como uma ponte entre a descoberta
de processos e a simulação de eventos discretos, traduzindo dados
empíricos em parâmetros quantitativos para o modelo. Essa integração
garante que a simulação gerada preserve as características
estatísticas do processo original, permitindo a avaliação confiável
de métricas de desempenho e a experimentação de cenários alternativos
de operação.

\section{Métricas de Qualidade de Modelos}

A qualidade dos modelos descobertos por meio da mineração de
processos é um fator determinante para a confiabilidade das análises
subsequentes e, principalmente, para o uso desses modelos como base
para simulações. A avaliação sistemática da qualidade garante que o
modelo minerado não apenas represente corretamente o comportamento
histórico, mas também seja capaz de generalizar o comportamento
futuro e sustentar decisões operacionais baseadas em evidências.

De acordo com van der Aalst \cite{aalst2016process}, um modelo de
processo deve ser avaliado em múltiplas dimensões de qualidade, de
modo a equilibrar precisão, generalização e simplicidade. As
principais métricas utilizadas na literatura são: \textit{fitness},
\textit{precision}, \textit{generalization} e \textit{simplicity}.
Essas métricas, amplamente adotadas em ferramentas como o
\textit{ProM Framework} e a \texttt{PM4Py}, compõem a base dos
algoritmos de verificação de conformidade (\textit{conformance
	checking}), responsáveis por comparar o comportamento observado nos
logs de eventos com o comportamento previsto pelo modelo minerado.

\begin{itemize}
	\item \textbf{Fitness}: avalia o quanto o modelo reproduz o comportamento observado no log. Um modelo com alto \textit{fitness} consegue reproduzir todas as sequências válidas de atividades sem apresentar falhas de execução.
	\item \textbf{Precision}: mede o nível de restrição do modelo, penalizando comportamentos não observados nos dados. Modelos excessivamente permissivos tendem a apresentar alta flexibilidade, mas baixa precisão.
	\item \textbf{Generalization}: representa a capacidade do modelo de capturar variações plausíveis do processo, evitando o sobreajuste aos dados históricos.
	\item \textbf{Simplicity}: reflete o grau de complexidade estrutural do modelo; modelos excessivamente complexos, embora precisos, dificultam a interpretação e a manutenção \cite{manifesto2011}.
\end{itemize}

O \textit{Process Mining Manifesto} \cite{manifesto2011} destaca a
importância de equilibrar essas dimensões, pois a busca por um modelo
com \textit{fitness} perfeito pode levar à perda de generalização, e
vice-versa. A maturidade da área de mineração de processos se reflete
justamente na capacidade de lidar com esse compromisso entre precisão
e abstração. Esse equilíbrio é particularmente relevante quando os
modelos descobertos serão empregados em simulações, uma vez que
comportamentos não observados ou superajustados podem comprometer a
validade dos resultados simulados.

Liu \cite{liu2015integrating} ressalta que a integração entre
mineração e simulação exige não apenas um modelo logicamente
consistente, mas também estatisticamente representativo. A validação
entre logs reais e logs sintéticos simulados permite medir a
aderência entre ambos, utilizando métricas de distância e
similaridade, como a \textit{edit distance}. Essa abordagem garante
que a simulação não apenas reproduza o fluxo de atividades, mas
também mantenha coerência temporal e probabilística com o processo
original.

Kherbouche et al. \cite{betterlogs2020} complementam essa perspectiva
ao argumentar que a qualidade do modelo está diretamente relacionada
à qualidade do log de eventos. Logs incompletos, redundantes ou
ruidosos geram modelos com baixa precisão e menor confiabilidade. Por
isso, o controle da qualidade dos logs — avaliado por dimensões como
\textit{completude}, \textit{consistência}, \textit{acurácia} e
\textit{complexidade} — é pré-requisito para a obtenção de métricas
significativas de \textit{fitness} e \textit{precision}.

Portanto, as métricas de qualidade de modelos constituem um elo
crítico entre mineração e simulação. Elas fornecem os indicadores
necessários para validar a representatividade, robustez e
aplicabilidade do modelo descoberto. Um modelo de processo só é
efetivamente útil quando combina boa aderência aos dados históricos
com capacidade de generalização para novos cenários — característica
essencial para o uso em ambientes de tomada de decisão e otimização
operacional.

\section{Geração de Logs Sintéticos}

A geração de logs sintéticos é uma etapa estratégica na integração
entre mineração de processos e simulação de eventos discretos. Essa
técnica permite criar registros artificiais de eventos que preservam
as propriedades estruturais e estatísticas de logs reais,
viabilizando experimentos controlados, testes de desempenho e
validação de modelos sem a necessidade de utilizar dados sensíveis ou
restritos. Em termos práticos, os logs sintéticos servem como uma
representação simulada do comportamento observado, permitindo avaliar
a fidelidade dos modelos minerados e a confiabilidade das previsões
operacionais.

Segundo van der Aalst \cite{aalst2016process}, a utilização de logs
artificiais é essencial para verificar se o modelo de processo
descoberto é capaz de reproduzir o comportamento do sistema real.
Essa comparação é comumente feita por meio de técnicas de
\textit{conformance checking}, nas quais o log sintético gerado pelo
modelo é confrontado com o log original, medindo-se métricas como
\textit{fitness} e \textit{precision}. Essa abordagem é
particularmente útil em ambientes dinâmicos, onde a estrutura do
processo pode mudar com frequência, como na área hospitalar e em
sistemas logísticos.

Ferronato \cite{ferronato2021pm2sim} propõe no framework
\textit{PM2Sim} um processo automatizado de geração de logs
sintéticos a partir de modelos descobertos via mineração de
processos. O sistema transforma o modelo minerado — geralmente
representado como uma Rede de Petri — em um modelo de simulação
implementado em Python por meio da biblioteca \texttt{SimPy}. Durante
a execução da simulação, eventos são registrados em formato XES
(\textit{eXtensible Event Stream}), mantendo a compatibilidade com
ferramentas de mineração como PM4Py e ProM. Essa estratégia permite a
retroalimentação do ciclo de mineração, criando uma integração
contínua entre descoberta, simulação e validação.

Augusto et al. \cite{augusto2016evaluation} reforçam essa importância
ao aplicar a integração entre mineração de processos e simulação em
um contexto clínico. Os autores desenvolveram uma metodologia para
simular fluxos de pacientes com base em logs hospitalares nacionais,
combinando mineração e simulação. Essa abordagem permitiu a criação
de logs sintéticos de trajetórias clínicas, utilizados para avaliar o
impacto de decisões médicas e políticas de gestão sobre taxas de
mortalidade e custos. Os resultados evidenciam o potencial da geração
de logs sintéticos como ferramenta de experimentação em ambientes de
alta complexidade e variabilidade.

A geração de dados artificiais também desempenha um papel importante
na análise de eficiência operacional de ambientes hospitalares. O
estudo de Protil et al. \cite{protil2004taxa} sobre a ocupação de
centros cirúrgicos demonstrou que o uso de modelos simulados
possibilita identificar desperdícios de tempo e gargalos no
agendamento de cirurgias. Embora o trabalho não empregue mineração de
processos, ele antecipa a importância da simulação como meio de
geração de dados para planejamento e otimização de recursos — uma
função equivalente à dos logs sintéticos em contextos modernos.

Outro aspecto relevante é a qualidade dos logs gerados. Conforme
Kherbouche et al. \cite{betterlogs2020}, a utilidade dos logs
sintéticos depende da fidelidade com que reproduzem as
características estatísticas, temporais e comportamentais dos
registros reais. Logs incompletos, redundantes ou inconsistentes
podem comprometer a avaliação do modelo.

A mineração fornece o modelo descritivo; a simulação, o ambiente de
experimentação; e os logs sintéticos, o instrumento de validação.
Esse ciclo permite aprimorar continuamente o modelo de processo,
ajustando suas propriedades comportamentais e estatísticas com base
em dados observados e simulados. Resumidamente, a geração de logs
sintéticos transforma o modelo minerado em uma ferramenta viva de
aprendizado e decisão, capaz de antecipar cenários e apoiar a
otimização operacional em tempo reduzido.

\section{Indicadores de Desempenho em Simulação}

A avaliação de desempenho é um componente essencial em sistemas que utilizam simulação de eventos discretos para otimização de processos. Em simulação, indicadores tradicionais incluem tempo de ciclo (\textit{lead time}), utilização de recursos, capacidade (\textit{throughput}) e eficiência operacional. Essas métricas permitem comparar cenários alternativos e validar se mudanças estruturais geram os ganhos esperados.

No contexto hospitalar específico, Souza, Vaccaro e Lima \cite{souza2020ore} propuseram o indicador \textit{Operating Room Effectiveness} (ORE), adaptado do conceito de \textit{Overall Equipment Effectiveness} (OEE) da manufatura enxuta. O ORE é calculado pelo produto de três dimensões — Disponibilidade, Desempenho e Qualidade — permitindo decompor perdas operacionais em categorias mensuráveis: falhas de equipamento, setup, variações de tempo, cancelamentos e reintervenções.

O ORE foi concebido para fornecer uma visão integrada da efetividade
operacional de salas cirúrgicas, permitindo identificar e classificar
perdas operacionais de forma sistemática. Diferentemente de
indicadores isolados como taxa de ocupação, tempo de cancelamentos ou
resultados financeiros, o ORE oferece uma métrica global que
correlaciona múltiplas dimensões do desempenho operacional.

A base conceitual do ORE provém do OEE, desenvolvido por Nakajima
\cite{nakajima1988introduction} no contexto da Manutenção Produtiva
Total (TPM). O OEE é calculado pelo produto de três índices:
Disponibilidade, Desempenho e Qualidade. Essa estrutura
multiplicativa permite identificar onde ocorrem as principais perdas
do sistema produtivo. A adaptação para o contexto hospitalar mantém
essa estrutura, mas reinterpreta as perdas considerando as
especificidades do fluxo cirúrgico.

A fórmula geral do ORE é expressa por:

\begin{equation}
	\text{ORE} = \text{Disponibilidade} \times \text{Desempenho} \times \text{Qualidade}
\end{equation}

Alternativamente, o ORE pode ser calculado como:

\begin{equation}
	\text{ORE} = \frac{\text{Tempo Total de Valor Agregado (TTAV)}}{\text{Tempo Total Disponível (TTA)}}
\end{equation}

Cada componente do indicador é detalhado nas subseções seguintes.

\subsection{Dimensões e Perdas Operacionais}

O ORE classifica as perdas operacionais em três classes principais,
totalizando sete tipos distintos de perdas. Essa classificação é
fundamental para direcionar ações de melhoria de forma precisa e
mensurável.

\subsubsection{Perdas de Disponibilidade}

A Disponibilidade mede a fração do tempo programado que efetivamente
foi utilizado para atividades cirúrgicas, descontando perdas
relacionadas ao planejamento e preparação das salas. É calculada
como:

\begin{equation}
	\text{Disponibilidade} = \frac{\text{Tempo Total Programado (TTS)}}{\text{Tempo Total Disponível (TTA)}}
\end{equation}

As perdas de disponibilidade incluem:

\begin{itemize}
	\item \textbf{Falha de Equipamento}: tempo de inatividade causado
	      por quebras ou manutenções não programadas de equipamentos cirúrgicos,
	      anestésicos ou de suporte.

	\item \textbf{Setup e Paradas Programadas}: tempo necessário para
	      limpeza, preparação e esterilização da sala entre procedimentos. Este
	      tempo é estrutural e inevitável, mas pode ser otimizado através de
	      padronização e melhoria de processos.

	\item \textbf{Não Agendamento}: tempo ocioso em turnos reservados a
	      cirurgiões específicos, mas sem cirurgias programadas. Esta perda é
	      particularmente relevante em hospitais universitários onde há
	      alocação pré-definida de blocos cirúrgicos por equipe médica.
\end{itemize}

\subsubsection{Perdas de Desempenho}

O Desempenho mede a fração do tempo programado que foi efetivamente
utilizado, descontando variações em relação ao planejado e
cancelamentos. É calculado como:

\begin{equation}
	\text{Desempenho} = \frac{\text{Tempo Total Utilizado (TTU)}}{\text{Tempo Total Programado (TTS)}}
\end{equation}

As perdas de desempenho incluem:

\begin{itemize}
	\item \textbf{Pequenas Paradas}: interrupções momentâneas durante o
	      procedimento cirúrgico causadas por problemas com equipamentos,
	      suprimentos, falta de materiais ou quedas de energia.

	\item \textbf{Variação no Tempo Cirúrgico}: diferenças entre o
	      tempo planejado e o tempo real de execução do procedimento. Cirurgias
	      que excedem significativamente o tempo previsto podem causar
	      cancelamentos em cascata das cirurgias subsequentes.

	\item \textbf{Cancelamentos}: cirurgias canceladas após o
	      agendamento resultam em capacidade ociosa não recuperável. Cerca de 50\% dos cancelamentos ocorrem por falta de
	      condição clínica do paciente ou absenteísmo, indicando falhas na
	      gestão do fluxo pré-operatório.
\end{itemize}

\subsubsection{Perdas de Qualidade}

A Qualidade mede a fração do tempo utilizado que efetivamente agregou
valor ao paciente, descontando retrabalhos e reintervenções. É
calculada como:

\begin{equation}
	\text{Qualidade} = \frac{\text{Tempo Total de Valor Agregado (TTAV)}}{\text{Tempo Total Utilizado (TTU)}}
\end{equation}

As perdas de qualidade incluem:

\begin{itemize}
	\item \textbf{Reintervenções Cirúrgicas}: procedimentos que
	      necessitaram ser refeitos devido a falhas técnicas, complicações
	      evitáveis ou erros no procedimento inicial. Este tipo de perda é
	      difícil de mensurar devido à complexidade de classificar
	      reintervenções como evitáveis ou inerentes à evolução clínica do
	      paciente.
\end{itemize}

\subsection{ORE como Driver de Simulação}

O uso do ORE transcende a função de medição passiva de desempenho. No
contexto de simulação de processos, o ORE atua como um
\textbf{indicador direcionador} que permite:

\begin{enumerate}
	\item \textbf{Identificação de Oportunidades}: a decomposição do
	      ORE nas sete categorias de perdas permite priorizar quais aspectos do
	      processo apresentam maior potencial de ganho.

	\item \textbf{Definição de Metas}: gestores podem estabelecer metas
	      específicas de melhoria para cada dimensão do ORE (por exemplo,
	      "reduzir cancelamentos em 30\%" ou "reduzir tempo de setup em 5
	      minutos").

	\item \textbf{Simulação de Cenários}: modelos de simulação podem
	      ser parametrizados para alcançar valores-alvo de ORE, permitindo
	      avaliar quais configurações operacionais (alocação de recursos,
	      mudanças no cronograma, redução de variabilidade) são necessárias
	      para atingir as metas estabelecidas.

	\item \textbf{Ciclo de Melhoria Contínua}: a medição periódica do
	      ORE permite avaliar o impacto de intervenções e ajustar estratégias
	      de otimização de forma iterativa.
\end{enumerate}

Souza et al. \cite{souza2020ore} demonstraram que cenários
hipotéticos de melhoria, quando simulados, podem estimar com precisão
o impacto de intervenções antes de sua implementação. Por exemplo, a
redução de 50\% nos cancelamentos combinada com eliminação de tempos
ociosos por não agendamento resultou em um ganho projetado de 31,6\%
no ORE, equivalente a aproximadamente 400 horas adicionais mensais de
capacidade cirúrgica.

Os resultados apresentados por Souza et al. \cite{souza2020ore}
demonstraram ganhos de eficiência de 12\% e economias anuais
estimadas em US\$400.000 após a implementação do indicador e de ações
de melhoria em um hospital universitário brasileiro. Tais resultados
evidenciam o potencial do ORE como métrica de apoio à gestão
operacional e à tomada de decisão em ambientes de alta complexidade.

Em trabalhos anteriores, Protil et al. \cite{protil2004taxa} já
haviam explorado o uso de modelagem e simulação de sistemas para
analisar a taxa de ocupação de centros cirúrgicos, apontando a
importância da simulação como ferramenta para otimização de recursos
hospitalares. A integração entre indicadores como o ORE e abordagens
baseadas em mineração e simulação, conforme sugerido por Ferronato
\cite{ferronato2021pm2sim}, amplia a capacidade analítica desses
sistemas, permitindo correlacionar métricas de eficiência com o
comportamento real dos processos.

Dessa forma, o uso combinado de indicadores operacionais como o ORE e
modelos minerados fornece uma visão quantitativa e dinâmica da
eficiência hospitalar, permitindo avaliar e prever o impacto de
decisões sobre produtividade, custos e qualidade dos serviços de
saúde.

\section{Padrão XES (eXtensible Event Stream)}

O padrão \textit{eXtensible Event Stream} (XES) foi desenvolvido pela
\textit{IEEE Task Force on Process Mining} com o objetivo de
padronizar a representação de logs de eventos utilizados em mineração
de processos. Formalizado pela norma IEEE 1849-2016
\cite{ieeexes2010}, o XES define uma estrutura extensível e
interoperável que permite armazenar e trocar informações sobre
execuções de processos entre diferentes ferramentas e plataformas.

Cada log XES é composto por um conjunto de \textit{traces}, que
representam casos individuais de execução, e cada \textit{trace}
contém uma sequência ordenada de \textit{events}, correspondentes às
atividades executadas. Cada evento possui atributos obrigatórios —
como nome da atividade, identificador do caso e carimbo de tempo — e
opcionais, como recursos, custos ou anotações adicionais. Essa
estrutura hierárquica assegura a consistência semântica dos dados e
permite análises multi-perspectiva (fluxo, tempo e organização).

O Código~\ref{fig:xes-exemplo} apresenta um trecho simplificado de um
log XES que segue o padrão IEEE, ilustrando a estrutura de um
\textit{trace} (caso) com três eventos correspondentes a um processo
de atendimento hospitalar.

\begin{lstlisting}[language=XML, caption={Exemplo simplificado de log XES}, label={fig:xes-exemplo}]
<?xml version="1.0" encoding="UTF-8"?>
<log xes.version="1.0" xes.features="nested-attributes"
     xmlns="http://www.xes-standard.org/">
  <trace>
    <string key="concept:name" value="Case001"/>
    <event>
      <string key="concept:name" value="Admissao do Paciente"/>
      <date key="time:timestamp" value="2024-03-15T08:30:00.000+00:00"/>
      <string key="org:resource" value="Recepcao"/>
    </event>
    <event>
      <string key="concept:name" value="Avaliacao Medica"/>
      <date key="time:timestamp" value="2024-03-15T09:00:00.000+00:00"/>
      <string key="org:resource" value="Dr. Silva"/>
    </event>
    <event>
      <string key="concept:name" value="Alta Hospitalar"/>
      <date key="time:timestamp" value="2024-03-15T10:15:00.000+00:00"/>
      <string key="org:resource" value="Administracao"/>
    </event>
  </trace>
</log>
\end{lstlisting}

Nesse exemplo, o caso \texttt{Case001} representa a trajetória de um
paciente desde a admissão até a alta hospitalar. Cada evento contém
informações sobre a atividade executada (\textit{concept:name}), o
horário em que ocorreu (\textit{time:timestamp}) e o recurso
responsável (\textit{org:resource}). A simplicidade e a
extensibilidade do formato permitem que diferentes sistemas coletem e
exportem dados compatíveis para posterior mineração.

% ----------------------------------------------------------
% Metodologia
% ----------------------------------------------------------

\chapter{Metodologia}

Este capítulo apresenta a metodologia utilizada para o
desenvolvimento do gerador de modelos de simulação baseado em
mineração de processos. A abordagem metodológica foi estruturada em
quatro etapas principais: análise automática de logs, mineração de
processos, simulação de eventos discretos e validação de qualidade.

\section{Visão Geral da Abordagem Metodológica}

A metodologia proposta consiste em um pipeline sequencial de quatro
etapas interdependentes, conforme ilustrado na Figura

\begin{enumerate}
	\item \textbf{Análise Automática de Logs}: detecção de atributos e validação de compatibilidade
	\item \textbf{Mineração de Processos}: extração do modelo formal e parâmetros estatísticos
	\item \textbf{Simulação de Logs Sintéticos}: geração de casos baseada em eventos discretos
	\item \textbf{Validação de Qualidade}: avaliação de similaridade e conformidade
\end{enumerate}

\section{Fluxo de Dados Detalhado}

A Figura~\ref{fig:fluxo_metodologia} ilustra o fluxo completo de
dados através do sistema:

\begin{figure}[htb]
	\centering
	\includegraphics[width=\textwidth,height=0.85\textheight,keepaspectratio]{figuras/fig2.png}
	\caption{Fluxo completo da metodologia. Fonte: Autor (2025)}
	\label{fig:fluxo_metodologia}
\end{figure}
\clearpage

Cada etapa produz artefatos específicos que alimentam a etapa
seguinte, garantindo rastreabilidade e reprodutibilidade do processo.
O sistema foi projetado para ser 100\% genérico, funcionando com logs
de qualquer domínio organizacional que disponha de registros
estruturados em formato XES.

\section{Etapa 1: Análise Automática de Logs}

\subsection{Diagrama de Componente}

A Figura~\ref{fig:fase1_componente} apresenta o diagrama de
componente da Etapa 1. O componente central, \texttt{LogAnalyzer}, é
responsável por realizar a análise automática e a extração de perfil
de um log de eventos.

O componente recebe como entrada um \textbf{log de eventos} no
formato XES, que encapsula os traces, eventos, atributos, timestamps
e recursos do processo. A análise é parametrizada por uma
\textbf{configuração} que define o modo de operação, como a ativação
da auto-detecção de atributos e a validação do formato.

Internamente, o \texttt{LogAnalyzer} utiliza \textbf{métodos de
	análise} baseados em bibliotecas como PM4Py para realizar a detecção
de padrões, análise de variantes e análise estatística. As principais
tarefas executadas são a detecção automática da estrutura do log, a
identificação das variantes do processo, a validação de
compatibilidade com as etapas subsequentes e a extração de
estatísticas descritivas.

Como resultado, o componente produz um \textbf{Perfil do Log}
(\texttt{LogProfile}), um artefato estruturado que contém as
características detectadas. Isso inclui métricas quantitativas
(número de casos, eventos, atividades e recursos únicos), estruturais
(variantes de processo, nomes dos atributos-chave identificados) e
temporais (estatísticas de duração e frequência), além de uma
verificação de compatibilidade para o pipeline.

\begin{figure}[htb]
	\centering
	\makebox[\textwidth][c]{\includegraphics[width=1.25\textwidth, keepaspectratio]{figuras/diagrama1.png}}
	\caption{Fase 1: Análise Automática de Logs. Fonte: Autor (2025)}
	\label{fig:fase1_componente}
\end{figure}

\subsection{Objetivo}

Detectar automaticamente as características estruturais, temporais e
operacionais do log de entrada, incluindo o cálculo de indicadores de
eficiência operacional quando aplicável, assegurando compatibilidade
com diferentes domínios e formatos sem necessidade de configuração
manual.

\subsection{Justificativa}

Logs de eventos variam significativamente entre domínios quanto à
nomenclatura de atributos, presença de recursos organizacionais e
convenções de registro temporal. A análise automática elimina a
necessidade de parametrização manual e permite processamento
agnóstico ao domínio, ampliando a aplicabilidade do sistema. Além da
análise estrutural, a integração de indicadores de eficiência como o
ORE permite quantificar perdas operacionais e direcionar a simulação
para cenários de melhoria específicos.

\subsection{Detecção de Atributos-Chave}

O sistema implementa um mecanismo de detecção baseado em prioridades,
testando sequencialmente diferentes convenções de nomenclatura até
identificar os atributos obrigatórios:

\begin{itemize}
	\item \textbf{Activity Key} (nome da atividade): \texttt{concept:name}, \texttt{Activity}, \texttt{activity}, \texttt{event}, \texttt{task}
	\item \textbf{Timestamp Key} (momento do evento): \texttt{time:timestamp}, \texttt{timestamp}, \texttt{Time}, \texttt{start\_time}
	\item \textbf{Case ID Key} (identificador do caso): \texttt{concept:name}, \texttt{case\_id}, \texttt{CaseID}, \texttt{Case}
	\item \textbf{Resource Key} (executor - opcional): \texttt{org:resource}, \texttt{resource}, \texttt{user}, \texttt{actor}
\end{itemize}

A ordem de prioridade baseia-se no padrão IEEE XES \cite{ieeexes2010}
e em análise de datasets públicos do BPI Challenge e repositório 4TU.

Quando nenhum candidato da lista de prioridades é encontrado, o
sistema aplica busca por palavras-chave nos nomes dos atributos. Se
ainda assim a detecção falhar para atributos obrigatórios, uma
exceção é lançada com lista dos atributos disponíveis para
diagnóstico.

\subsection{Coleta de Estatísticas Estruturais}

Para cada log analisado, o sistema extrai as seguintes informações:

\begin{itemize}
	\item Número total de casos (traces) e eventos
	\item Número de atividades únicas
	\item Distribuição de frequência das atividades
	\item Comprimento mínimo, máximo e médio dos traces
	\item Número de variantes do processo (sequências únicas de atividades)
	\item Distribuição de recursos por atividade (quando disponível)
\end{itemize}

Essas estatísticas são armazenadas em um objeto \texttt{LogProfile},
que serve como entrada para as etapas posteriores e permite análise
exploratória dos dados antes da mineração.

\subsection{Análise de Eficiência Operacional (ORE)}

Quando o log de eventos contém informações suficientes sobre
planejamento e execução de atividades, o sistema realiza
automaticamente o cálculo de indicadores de eficiência operacional
baseados no modelo ORE proposto por Souza et al. \cite{souza2020ore}.
Esta análise é particularmente relevante para processos hospitalares,
mas pode ser aplicada a qualquer domínio que possua dados de
agendamento, execução e resultados de atividades.

\subsubsection{Cálculo de Métricas ORE}

O cálculo do ORE é mais preciso quando o log de eventos é enriquecido
com atributos operacionais específicos. O sistema é capaz de
processar tanto logs padrão, inferindo perdas, quanto logs
enriquecidos, utilizando dados reais para maior acurácia. Um
\textbf{log XES enriquecido}, neste contexto, pode conter os
seguintes atributos adicionais em nível de caso:

\begin{itemize}
	\item \texttt{custom:status}: status da atividade (ex: \textit{Realizada}, \textit{Cancelada}). Permite a contagem real de cancelamentos.
	\item \texttt{custom:actual\_duration}: duração real da atividade. Permite cálculo preciso de perdas de desempenho.
	\item \texttt{custom:cancellation\_reason}: motivo do cancelamento. Permite análise de causa raiz.
\end{itemize}

Quando esses atributos não estão presentes, o sistema busca por
atributos candidatos genéricos para estimar as métricas:

\begin{itemize}
	\item \textbf{Tempo Planejado}: duração prevista para a atividade
	      (atributos candidatos: \texttt{planned\_duration},
	      \texttt{scheduled\_time}, \texttt{expected\_duration})

	\item \textbf{Tempo Disponível}: janela temporal alocada para a
	      execução (atributos candidatos: \texttt{available\_time},
	      \texttt{slot\_duration}, \texttt{scheduled\_slot})

	\item \textbf{Status de Execução}: indicador de cancelamentos ou
	      falhas (atributos candidatos: \texttt{status}, \texttt{outcome},
	      \texttt{execution\_status})

	\item \textbf{Tipo de Perda}: classificação da perda operacional
	      quando aplicável (atributos candidatos: \texttt{loss\_type},
	      \texttt{waste\_category}, \texttt{failure\_reason})
\end{itemize}

Quando esses atributos estão presentes, o sistema calcula as três
dimensões do ORE:

\begin{enumerate}
	\item \textbf{Disponibilidade}: razão entre o tempo total programado
	      (TTS) e o tempo total disponível (TTA), descontando perdas por falha
	      de equipamento, setup e não agendamento.

	\item \textbf{Desempenho}: razão entre o tempo total utilizado (TTU)
	      e o tempo total programado (TTS), descontando perdas por pequenas
	      paradas, variação no tempo cirúrgico e cancelamentos.

	\item \textbf{Qualidade}: razão entre o tempo total de valor
	      agregado (TTAV) e o tempo total utilizado (TTU), descontando perdas
	      por reintervenções.
\end{enumerate}

O ORE global é calculado como o produto das três dimensões,
resultando em um indicador percentual que reflete a efetividade
global do processo.

\subsection{Saída Integrada da Etapa}

A saída da Etapa 1 foi expandida para incluir não apenas o perfil
estrutural do log, mas também indicadores operacionais quando
aplicável. A saída consiste em:

\begin{itemize}
	\item \textbf{LogProfile}: objeto contendo todos os metadados
	      detectados, incluindo mapeamento de atributos, estatísticas
	      estruturais e flags de compatibilidade.

	\item \textbf{OREMetrics} (opcional): objeto contendo métricas de
	      eficiência operacional, incluindo:
	      \begin{itemize}
		      \item Valores de Disponibilidade, Desempenho e Qualidade
		      \item ORE global
		      \item Decomposição granular das sete categorias de perdas
	      \end{itemize}
\end{itemize}

Quando o log não contém informações suficientes para cálculo do ORE,
o sistema prossegue apenas com o \texttt{LogProfile}, mantendo
compatibilidade com processos de qualquer domínio. A presença do
\texttt{OREMetrics} adiciona uma camada analítica que permite
direcionar a simulação (Etapa 3) para cenários de otimização
específicos.

\section{Etapa 2: Mineração de Processos}

\subsection{Diagrama de Componente}

A Figura~\ref{fig:fase2_componente} apresenta o diagrama de
componente da Etapa 2, detalhando o processo de mineração de
processos e extração de parâmetros estatísticos.

O componente central \texttt{ProcessMiner} recebe como entradas o
arquivo XES original, o objeto \texttt{LogProfile} produzido na Etapa
1 contendo os metadados do log, o parâmetro \texttt{variant\_filter}
configurado em 0.8 para filtragem de variantes, e uma flag
\texttt{save\_model\_image} para geração opcional de visualização do
modelo. O processamento interno é organizado em cinco fases
sequenciais: (1) filtragem de variantes baseada em frequência,
mantendo apenas as variantes que representam 80\% dos casos totais
para remoção de ruído; (2) descoberta do modelo de processo
utilizando o algoritmo Inductive Miner, que garante propriedades de
soundness e sempre produz um modelo bem-formado; (3) conversão do
Process Tree resultante para uma Rede de Petri formal com marcação
inicial e final; (4) ajuste de distribuições estatísticas (Normal,
Log-Normal e Exponencial) às durações observadas de cada atividade,
utilizando o teste de Kolmogorov-Smirnov para seleção da distribuição
com melhor aderência; (5) avaliação de qualidade do modelo através
das métricas de fitness, precision e simplicity.

A saída produzida consiste em um objeto estruturado
\texttt{ProcessModel} contendo: a Rede de Petri formal representada
pela tupla (\texttt{net}, \texttt{im}, \texttt{fm}), onde
\texttt{net} é a rede, \texttt{im} é a marcação inicial e \texttt{fm}
é a marcação final; um dicionário de \texttt{ActivityStatistics} para
cada atividade, incluindo duração média, desvio padrão, distribuição
ajustada e seus parâmetros; métricas temporais globais como
\texttt{arrival\_rate} (taxa de chegada de casos),
\texttt{dispersion\_rate} e \texttt{median\_duration}; métricas de
qualidade do modelo (\texttt{fitness}, \texttt{precision},
\texttt{simplicity}); mapeamento de recursos por atividade quando
disponível; e o \texttt{LogProfile} original para rastreabilidade. As
bibliotecas PM4Py são utilizadas para implementação do Inductive
Miner e cálculo de métricas de conformidade, enquanto SciPy (módulo
\texttt{scipy.stats}) é empregado para ajuste e avaliação das
distribuições estatísticas.

\begin{figure}[htb]
	\centering
	\makebox[\textwidth][c]{\includegraphics[width=1.25\textwidth, keepaspectratio]{figuras/diagrama2.png}}
	\caption{Fase 2: Mineração de Processos. Fonte: Autor (2025)}
	\label{fig:fase2_componente}
\end{figure}

\subsection{Objetivo}

Extrair o modelo formal do processo na forma de uma Rede de Petri e
parametrizar suas características temporais e organizacionais para
uso na simulação.

\subsection{Filtragem de Variantes}

\subsubsection{Problema}

Logs reais frequentemente contêm ruído, casos excepcionais e
variantes raras que dificultam a descoberta de modelos
representativos e podem levar a overfitting.

\subsubsection{Solução}

Aplicação de filtragem baseada em frequência, mantendo apenas as
variantes que representam um percentual especificado dos casos
totais.

\subsubsection{Parâmetro Padrão}

O sistema utiliza como padrão a retenção de 80\% das variantes mais
frequentes (\texttt{variant\_filter=0.8}).

\subsubsection{Justificativa}

O valor de 80\% fundamenta-se no Princípio de Pareto, onde
aproximadamente 20\% das variantes explicam 80\% do comportamento
observado \cite{aalst2016process}. Este limiar oferece balance
adequado entre cobertura comportamental e remoção de ruído.

\subsection{Descoberta do Modelo de Processo}

\subsubsection{Algoritmo Selecionado}

O sistema utiliza o \textbf{Inductive Miner} proposto por Leemans,
Fahland e van der Aalst \cite{leemans2013discovering}.

\subsubsection{Justificativa da Escolha}

A Tabela~\ref{tab:algoritmos} apresenta comparação entre algoritmos
de descoberta disponíveis:

\begin{table}[htb]
	\centering
	\caption{Comparação entre algoritmos de descoberta de processos}
	\label{tab:algoritmos}
	\begin{tabular}{l>{\raggedright\arraybackslash}p{4cm}>{\raggedright\arraybackslash}p{4cm}>{\raggedright\arraybackslash}p{3cm}}
		\hline
		\textbf{Algoritmo}       & \textbf{Vantagens}                                         & \textbf{Desvantagens}     & \textbf{Decisão}   \\
		\hline
		Alpha Miner              & Simples, pioneiro                                          & Não lida com ruído, loops & Não escolhido      \\
		Heuristic Miner          & Tolerante a ruído                                          & Modelos não-formais       & Não escolhido      \\
		\textbf{Inductive Miner} & Soundness garantido, robusto a ruído, sempre produz modelo & Pode generalizar demais   & \textbf{ESCOLHIDO} \\
		Split Miner              & Alta precisão                                              & Requer tuning complexo    & Não escolhido      \\
		\hline
	\end{tabular}
\end{table}

O Inductive Miner foi escolhido por garantir três propriedades
fundamentais:

\begin{enumerate}
	\item \textbf{Soundness}: modelo sempre bem-formado, sem deadlocks
	\item \textbf{Fitness}: modelo sempre capaz de reproduzir o log
	\item \textbf{Completude}: sempre produz um modelo, mesmo com logs problemáticos
\end{enumerate}

Além disso, o algoritmo gera modelos estruturados em blocos
(block-structured), facilitando a conversão para formatos simuláveis.

\subsubsection{Saída}

Uma Rede de Petri formalmente definida como a tupla $N = (P, T, F)$,
onde:
\begin{itemize}
	\item $P$: conjunto de lugares (places)
	\item $T$: conjunto de transições (transitions)
	\item $F \subseteq (P \times T) \cup (T \times P)$: conjunto de arcos
\end{itemize}

Acompanhada de marcação inicial ($M_0$) e marcação final ($M_f$).

\subsection{Extração de Estatísticas Temporais}

\subsubsection{Cálculo de Durações}

Para cada atividade, o sistema calcula durações baseando-se em:

\textbf{Casos com timestamp de conclusão}:
\begin{equation}
	duration = time:complete - time:timestamp
\end{equation}

\textbf{Casos sem timestamp de conclusão} (maioria dos logs):
\begin{equation}
	duration = timestamp(event_{i+1}) - timestamp(event_i)
\end{equation}

\subsection{Ajuste de Distribuições Estatísticas}

\subsubsection{Objetivo}

Modelar a variabilidade realista das durações para simulação
estocástica fiel ao comportamento observado.

\subsubsection{Distribuições Candidatas}

O sistema testa três distribuições de probabilidade:

\begin{enumerate}
	\item \textbf{Normal (Gaussiana)}: para atividades com duração relativamente constante e variação simétrica
	\item \textbf{Log-Normal}: para atividades com cauda longa à direita, comum em processos humanos
	\item \textbf{Exponencial}: para tempos de espera e processos de chegada
\end{enumerate}

\subsection{Avaliação de Qualidade do Modelo}

O sistema calcula três métricas complementares de qualidade
\cite{aalst2016process}:

\subsubsection{Fitness (0-1, maior é melhor)}

\textbf{Definição}: proporção de comportamento do log que o modelo consegue reproduzir.

\textbf{Método}: replay fitness baseado em alinhamentos.

\textbf{Cálculo}: O fitness utiliza token-based replay ou alignment-based conformance. A fórmula mais comum é:

\begin{equation}
	\text{fitness} = 1 - \frac{\text{custo\_total}}{\text{custo\_máximo\_possível}}
\end{equation}

\textbf{Definição do Custo}: O custo representa o número de operações necessárias para fazer o modelo aceitar um trace do log. Durante a mineração, o custo inclui: (1) \textit{movimentos no modelo} - transições silenciosas que precisam ser executadas, (2) \textit{movimentos no log} - eventos que precisam ser ignorados, e (3) \textit{movimentos síncronos} - eventos que coincidem perfeitamente (custo zero). Algoritmos de mineração usam esse custo para decidir quando parar de dividir o log e qual divisão produz o melhor modelo.

No contexto de alinhamentos, para cada trace individual:
\begin{equation}
	\text{fitness\_trace}_i = 1 - \frac{\text{edit\_distance}_i}{\text{max\_length}_i}
\end{equation}

O fitness global do log é calculado como:
\begin{equation}
	\text{fitness\_log} = \frac{1}{n} \sum_{i=1}^{n} \text{fitness\_trace}_i
\end{equation}

onde $n$ é o número total de traces no log. O PM4Py implementa
automaticamente esses cálculos utilizando algoritmos de alinhamento
otimizados.

\textbf{Interpretação}:
\begin{itemize}
	\item $\geq 0.9$: modelo explica quase todo o log
	\item $0.7-0.9$: modelo explica maior parte
	\item $< 0.7$: modelo inadequado
\end{itemize}

\subsubsection{Precision (0-1, maior é melhor)}

\textbf{Definição}: quão preciso é o modelo, evitando comportamento extra não observado.

\textbf{Método}: token-based conformance.

\textbf{Cálculo}: A precision mede quantos comportamentos extras (não observados) o modelo permite. Utiliza a fórmula ETC (Escaping Arcs):

\begin{equation}
	\text{precision} = 1 - \frac{\text{arcos\_escapando}}{\text{arcos\_totais}}
\end{equation}

onde os arcos escapando representam transições que podem ser
executadas pelo modelo mas não foram observadas no log original. O
PM4Py calcula automaticamente essa métrica através de algoritmos de
conformance checking que identificam comportamentos não observados.

\textbf{Interpretação}:
\begin{itemize}
	\item $\geq 0.8$: modelo preciso
	\item $< 0.5$: modelo muito generalista
\end{itemize}

\subsubsection{Simplicity (0-1, maior é melhor)}

\textbf{Definição}: simplicidade estrutural do modelo.

\textbf{Método}: razão entre arcos e nós na rede de Petri.

\textbf{Cálculo}: A simplicidade mede a complexidade topológica da Rede de Petri através da fórmula estrutural:

\begin{equation}
	\text{simplicity} = \frac{1}{1 + \frac{|\text{arcos}|}{|\text{nós}|}}
\end{equation}

onde $|\text{arcos}|$ representa o número total de arcos (transições)
e $|\text{nós}|$ representa o número total de nós (places) na rede de
Petri. O PM4Py calcula automaticamente essa métrica analisando a
estrutura topológica do modelo descoberto.

\textbf{Interpretação}:
\begin{itemize}
	\item $\geq 0.7$: modelo simples
	\item $< 0.4$: modelo complexo
\end{itemize}

\subsection{Saída da Etapa}

\noindent Um objeto \texttt{ProcessModel} contendo:
\begin{itemize}
	\item Rede de Petri (net, im, fm)
	\item Dicionário de estatísticas de atividades
	      (\texttt{ActivityStatistics})
	\item Métricas globais (arrival\_rate, dispersion\_rate, median\_duration)
	\item Métricas de qualidade (fitness, precision, simplicity)
	\item Mapeamento de recursos (opcional)
	\item Perfil do log original (\texttt{LogProfile})
\end{itemize}

\section{Etapa 3: Simulação de Logs Sintéticos}

\subsection{Diagrama de Componente}

A Figura~\ref{fig:fase3_componente} apresenta o diagrama de
componente da Etapa 3, ilustrando o processo de simulação de logs
sintéticos baseado em eventos discretos.

O componente central \texttt{LogSimulator} recebe como entradas o
objeto \texttt{ProcessModel} contendo a Rede de Petri e todos os
parâmetros estatísticos extraídos na Etapa 2, e um objeto
\texttt{SimulationConfig} que especifica os parâmetros da simulação:
\texttt{num\_cases} (padrão 100) para número de casos a serem
gerados, \texttt{arrival\_rate} (extraído do modelo) para taxa de
chegada de novos casos, \texttt{activity\_durations} (extraídos do
modelo) contendo as distribuições estatísticas de duração de cada
atividade, e \texttt{random\_seed} (padrão 42) para garantir
reprodutibilidade dos experimentos.

O processamento interno implementa um simulador de eventos discretos
utilizando a biblioteca SimPy, organizado em quatro componentes
principais: (1) gerador de chegadas, que cria novos casos seguindo
uma distribuição exponencial baseada na \texttt{arrival\_rate}
observada; (2) executor de casos, que para cada caso individual
percorre a Rede de Petri respeitando sua semântica formal,
identificando transições habilitadas e escolhendo uma aleatoriamente
para disparo; (3) gerador de eventos, que para cada transição
disparada cria um evento no log sintético com \texttt{case\_id},
\texttt{activity}, \texttt{timestamp} (tempo atual da simulação) e
\texttt{resource} (selecionado aleatoriamente do pool de recursos
disponíveis para aquela atividade); (4) controlador de duração, que
para cada atividade amostra uma duração da distribuição estatística
ajustada (Normal, Log-Normal ou Exponencial) e avança o relógio da
simulação pelo tempo correspondente. A simulação continua até que
todos os casos alcancem a marcação final da Rede de Petri ou atinjam
o limite de segurança (\texttt{max\_trace\_length=1000}) para
proteção contra loops infinitos.

A saída produzida consiste em um objeto \texttt{SimulationResult}
contendo: caminho para o arquivo CSV intermediário com os eventos
gerados em formato tabular (\texttt{case\_id}, \texttt{activity},
\texttt{time:timestamp}, \texttt{resource}); caminho para o arquivo
XES final convertido seguindo o padrão IEEE 1849-2016, incluindo
classificador de atividade para compatibilidade com ferramentas como
ProM e Disco; estatísticas da simulação incluindo
\texttt{num\_cases\_generated}, \texttt{num\_events\_generated},
\texttt{simulation\_time\_seconds} e \texttt{generation\_timestamp}.
As bibliotecas SimPy são utilizadas para implementação da lógica de
eventos discretos (processos, timeouts, environment), enquanto PM4Py
é empregada para conversão do formato intermediário CSV para o
formato XES padronizado.

\begin{figure}[htb]
	\centering
	\makebox[\textwidth][c]{\includegraphics[width=1.25\textwidth, keepaspectratio]{figuras/diagrama3.png}}
	\caption{Fase 3: Simulação de Logs Sintéticos. Fonte: Autor (2025)}
	\label{fig:fase3_componente}
\end{figure}

\subsection{Objetivo}

Gerar novos casos de processo que sigam o modelo descoberto e as
distribuições estatísticas extraídas, preservando características
estruturais e temporais do processo original.

\subsection{Paradigma de Simulação}

\subsubsection{Abordagem Escolhida}

O sistema utiliza \textbf{Discrete Event Simulation (DES)}
implementada com a biblioteca SimPy.

\subsubsection{Justificativa}

\noindent DES é adequado porque:
\begin{itemize}
	\item Processos de negócio são inerentemente discretos (atividades têm
	      início e fim definidos)
	\item Eventos ocorrem em pontos específicos do tempo
	\item Estado do sistema muda apenas em eventos
	\item SimPy oferece abstrações adequadas (processos, timeouts, recursos)
\end{itemize}

\subsection{Configuração da Simulação}

A Tabela~\ref{tab:parametros} lista os parâmetros principais:

\begin{table}[htb]
	\centering
	\caption{Parâmetros de configuração da simulação}
	\label{tab:parametros}
	\begin{tabular}{llp{6cm}}
		\hline
		\textbf{Parâmetro}           & \textbf{Valor Padrão} & \textbf{Justificativa}                           \\
		\hline
		\texttt{num\_cases}          & 100                   & Suficiente para análise estatística sem overhead \\
		\texttt{arrival\_rate}       & Do modelo             & Preserva taxa de chegada original                \\
		\texttt{activity\_durations} & Do modelo             & Preserva durações originais                      \\
		\texttt{random\_seed}        & 42                    & Reprodutibilidade dos experimentos               \\
		\texttt{max\_trace\_length}  & 1000                  & Limite de segurança contra loops                 \\
		\hline
	\end{tabular}
\end{table}

Todos os parâmetros podem ser sobrescritos para simulação de cenários
alternativos (por exemplo, "E se reduzirmos durações em 50\%?").

\subsection{Geração de Casos}

\subsubsection{Processo de Chegadas}

O sistema implementa um gerador SimPy que cria casos sequencialmente:

\begin{verbatim}
Para cada caso i de 1 até num_cases:
    1. Aguardar intervalo de chegada (arrival_rate)
    2. Iniciar processo paralelo para simular caso i
\end{verbatim}

Este padrão permite múltiplos casos ativos simultaneamente,
representando carga de trabalho real com concorrência.

\subsection{Simulação Individual de Casos}

O sistema implementa algoritmo baseado na semântica formal de Redes
de Petri \cite{peterson1981petri}:

\begin{algorithm}[H]
	\caption{Simulação de um caso individual}
	\begin{algorithmic}[1]
		\State \textbf{Entrada:} Rede de Petri $(net, im, fm)$
		\State $marking \gets im$ \Comment{Marcação inicial}
		\While{$marking \neq fm$} \Comment{Até marcação final}
		\State $T_{enabled} \gets$ transições habilitadas em $marking$
		\If{$T_{enabled} = \emptyset$}
		\State \textbf{break} \Comment{Marcação final ou deadlock}
		\EndIf
		\State $t \gets$ escolhe aleatoriamente de $T_{enabled}$
		\If{$t$ não é transição silenciosa}
		\State $timestamp \gets \text{tempo\_atual\_simulação}$
		\State $recurso \gets$ escolhe aleatório de $\text{recursos}(t.\text{atividade})$
		\State Registrar evento$(caso\_id, t.atividade, timestamp, recurso)$
		\State Aguardar $\text{duração}(t.\text{atividade})$ segundos
		\EndIf
		\State $marking \gets \text{executa}(t, marking)$ \Comment{Atualizar marcação}
		\If{comprimento\_trace $> max\_trace\_length$}
		\State \textbf{break} \Comment{Limite de segurança}
		\EndIf
		\EndWhile
	\end{algorithmic}
\end{algorithm}
\subsection{Geração dos Logs de Saída}

\subsubsection{Formato Intermediário (CSV)}

\begin{verbatim}
case_id,activity,time:timestamp,resource
Case 1,Register Request,2024-10-13 10:00:00,John
Case 1,Examine,2024-10-13 10:02:30,Mary
...
\end{verbatim}

\subsubsection{Conversão para XES}

\noindent Processo automatizado:

\begin{enumerate}
	\item Leitura do CSV com Pandas
	\item Renomeação de colunas para padrão XES IEEE
	\item Formatação via \texttt{pm4py.format\_dataframe()}
	\item Conversão para estrutura de log PM4Py
	\item Exportação XES via \texttt{pm4py.write\_xes()}
	\item Inserção manual de classificador (workaround limitação PM4Py)
\end{enumerate}

O classificador XES inserido:
\begin{verbatim}
<classifier name="Activity" keys="concept:name"/>
\end{verbatim}

É necessário para compatibilidade com ferramentas como ProM e Disco.

\subsection{Saída da Etapa}

\noindent Um objeto \texttt{SimulationResult} contendo:
\begin{itemize}
	\item Caminhos dos arquivos (CSV e XES)
	\item Número de casos e eventos gerados
	\item Tempo de execução da simulação
	\item Timestamp de geração
\end{itemize}

\section{Etapa 4: Validação de Qualidade}

\subsection{Diagrama de Componente}

A Figura~\ref{fig:fase4_componente} apresenta o diagrama de
componente da Etapa 4, detalhando o processo de validação de
qualidade dos logs sintéticos gerados.

O componente central \texttt{LogValidator} recebe como entradas dois
arquivos XES: o arquivo \texttt{original\_xes} contendo o log de
eventos real e o arquivo \texttt{simulated\_xes} contendo o log
sintético gerado na Etapa 3.

O processamento interno implementa uma validação multidimensional que
compara diferentes perspectivas dos logs: (1) \textbf{Análise
	Estrutural}, onde calcula-se a similaridade dos traces usando
distância de edição; (2) \textbf{Análise de Frequência}, que compara
as distribuições de atividades e variantes; (3) \textbf{Análise
	Temporal}, que mede a similaridade das distribuições de duração das
atividades e aplica o teste de Kolmogorov-Smirnov para validação
estatística.

A saída produzida consiste em um objeto estruturado
\texttt{ValidationResult} contendo as métricas agregadas de todas as
análises, incluindo percentual de similaridade, p-valor e
estatísticas detalhadas. A biblioteca PM4Py é utilizada para os
cálculos de alinhamento e extração de características dos logs.

\begin{figure}[htb]
	\centering
	\makebox[\textwidth][c]{\includegraphics[width=1.25\textwidth, keepaspectratio]{figuras/diagrama4.png}}
	\caption{Fase 4: Validação de Qualidade. Fonte: Autor (2025)}
	\label{fig:fase4_componente}
\end{figure}

\subsection{Objetivo}

Avaliar quão similares são os logs original e sintético, validando a
qualidade da geração e a fidelidade do modelo. A validação é
multidimensional, comparando os logs sob perspectivas estrutural, de
frequência e temporal.

\subsection{Métricas de Validação}

O sistema calcula um conjunto de métricas para fornecer uma visão
holística da similaridade entre os logs.

\subsubsection{Similaridade Estrutural (Fitness de Alinhamento)}

A similaridade estrutural é calculada usando alinhamentos baseados em
\textbf{Distância de Edição} (\textit{Edit Distance}) entre os
traces. Para cada par de traces (um original, um sintético),
calcula-se o número mínimo de operações (inserção, deleção,
substituição) para transformar um no outro. A partir do custo do
alinhamento, deriva-se uma métrica de \textit{fitness} normalizada:

\begin{equation}
	fitness = 1 - \frac{\text{custo\_alinhamento}}{\text{custo\_máximo}}
\end{equation}

Onde $\text{custo\_máximo}$ é o custo do pior caso (deletar todos os
eventos de um trace e inserir todos do outro). O \textit{fitness}
médio de todos os alinhamentos resulta em um percentual de
similaridade estrutural.

\subsubsection{Similaridade de Distribuição de Atividades}

Esta métrica compara a frequência relativa de cada atividade entre o
log original e o sintético. Utiliza-se a distância de Jaccard ou
similar para quantificar a sobreposição das distribuições de
frequência. Um valor próximo de 1.0 indica que as atividades mais
comuns no processo real também são as mais comuns no processo
simulado.

\subsubsection{Similaridade de Distribuição de Variantes}

Compara a distribuição das variantes (sequências de atividades
únicas) entre os dois logs. Esta métrica avalia se o modelo simulado
consegue reproduzir os caminhos mais frequentes do processo real com
uma frequência parecida. Assim como na métrica de atividades, um
valor próximo de 1.0 é desejável.

\subsubsection{Similaridade de Distribuição de Durações}

Avalia se as durações das atividades no log sintético seguem uma
distribuição estatística similar às do log original. A comparação é
feita utilizando a distância de Wasserstein (Earth Mover's Distance)
ou outra métrica de distância entre distribuições, que mede o
"esforço" necessário para transformar uma distribuição na outra.

\subsubsection{Validação Estatística (Teste de Kolmogorov-Smirnov)}

Para uma validação mais rigorosa das durações, o sistema aplica o
teste de Kolmogorov-Smirnov (K-S) de duas amostras. Este teste não
paramétrico avalia a hipótese nula de que as amostras de duração
(original e sintética) foram extraídas da mesma distribuição. O
resultado é um \textbf{p-valor}:

\begin{itemize}
	\item \textbf{p-valor > 0.05}: Não há evidência estatística para rejeitar a hipótese nula. As distribuições são consideradas estatisticamente similares.
	\item \textbf{p-valor $\leq$ 0.05}: A hipótese nula é rejeitada. As distribuições são consideradas estatisticamente diferentes.
\end{itemize}

Um p-valor alto indica que a simulação reproduz fielmente a
variabilidade temporal do processo original.

\subsection{Agregação e Interpretação de Resultados}

As métricas são agregadas em um placar de validação. A
Tabela~\ref{tab:interpretacao} apresenta os thresholds para
interpretação.

\begin{table}[htb]
	\centering
	\caption{Interpretação de métricas de validação}
	\label{tab:interpretacao}
	\begin{tabular}{ll}
		\hline
		\textbf{Métrica}          & \textbf{Interpretação}                     \\
		\hline
		Similaridade Estrutural   & $\geq 80\%$ (Boa), $\geq 90\%$ (Excelente) \\
		Similaridade (Atividades) & $\geq 0.8$ (Boa)                           \\
		Similaridade (Variantes)  & $\geq 0.7$ (Boa)                           \\
		Similaridade (Durações)   & $\geq 0.8$ (Boa)                           \\
		KS-Test p-valor           & $> 0.05$ (Similaridade Estatística)        \\
		\hline
	\end{tabular}
\end{table}

\subsection{Saída da Etapa}

Um objeto \texttt{ValidationResult} contendo:
\begin{itemize}
	\item Percentual de similaridade estrutural (baseado em fitness médio)
	\item Similaridade de distribuição de atividades, variantes e durações
	\item P-valor do teste de Kolmogorov-Smirnov
	\item Detalhes estatísticos (médias, mínimos, máximos de fitness e custo)
\end{itemize}

\section{Parâmetros e Configurações}

\subsection{Tabela de Parâmetros Principais}

A Tabela~\ref{tab:parametros_completos} apresenta todos os parâmetros
configuráveis do sistema:

\begin{table}[htb]
	\centering
	\caption{Parâmetros configuráveis do sistema}
	\label{tab:parametros_completos}
	\begin{tabular}{lccl}
		\hline
		\textbf{Parâmetro}          & \textbf{Padrão} & \textbf{Faixa}   & \textbf{Impacto}                \\
		\hline
		\texttt{variant\_filter}    & 0.8             & 0.0-1.0          & Filtragem de ruído na mineração \\
		\texttt{num\_cases}         & 100             & 1-$\infty$       & Tamanho do log sintético        \\
		\texttt{arrival\_rate}      & Auto            & 0.1-$\infty$ min & Taxa de chegada de casos        \\
		\texttt{random\_seed}       & 42              & 0-$2^{32}$-1     & Reprodutibilidade               \\
		\texttt{max\_trace\_length} & 1000            & 1-$\infty$       & Proteção contra loops           \\
		\texttt{verbose}            & True            & True/False       & Saída de diagnóstico            \\
		\texttt{save\_model\_image} & None            & Path/None        & Visualização do modelo          \\
		\hline
	\end{tabular}
\end{table}

\section{Ferramentas e Tecnologias}

\subsection{Bibliotecas Principais}

A Tabela~\ref{tab:bibliotecas} apresenta as tecnologias utilizadas.

\begin{table}[htb]
	\centering
	\caption{Bibliotecas e ferramentas utilizadas}
	\label{tab:bibliotecas}
	\small
	\begin{tabular}{llp{6cm}}
		\hline
		\textbf{Biblioteca} & \textbf{Versão} & \textbf{Justificativa}                                                     \\
		\hline
		PM4Py               & 2.2.22          & Padrão de facto em Python para process mining, algoritmos state-of-the-art \\
		SimPy               & 4.0.1           & Leve, Pythônico, documentação excelente para DES                           \\
		SciPy               & 1.13.0          & Completa, bem testada, K-S test built-in                                   \\
		Pandas              & 2.2.2           & Eficiente para manipulação de dados, operações vetorizadas                 \\
		NumPy               & 1.26.4          & Base do ecossistema científico Python                                      \\
		Streamlit           & 1.28.0          & Interface web rápida e intuitiva                                           \\
		Graphviz            & 0.20.3          & Visualização de Redes de Petri                                             \\
		\hline
	\end{tabular}
\end{table}

\subsection{Justificativa das Escolhas}

A escolha das bibliotecas foi fundamentada em critérios técnicos
específicos para cada funcionalidade do sistema. O PM4Py representa o
padrão de facto em Python para process mining, oferecendo algoritmos
state-of-the-art com documentação completa. O SimPy foi selecionado
para simulação discreta de eventos por sua simplicidade e adequação
perfeita para modelagem de sistemas discretos. O SciPy foi escolhido
para análise estatística por sua robustez e confiabilidade.

Para manipulação de dados, o Pandas foi selecionado por sua
eficiência em operações vetorizadas, essencial para processamento dos
logs de eventos. O NumPy serve como base fundamental do ecossistema
científico Python, proporcionando operações matemáticas otimizadas. O
Streamlit foi escolhido para criar interfaces web rápidas e
intuitivas, facilitando a interação com o sistema. O Graphviz foi
selecionado especificamente para visualização de Redes de Petri,
oferecendo capacidades gráficas adequadas para representação dos
modelos de processo descobertos.

\section{Pseudocódigo de Alto Nível}

O algoritmo~\ref{alg:pipeline} apresenta o pipeline completo:

\begin{algorithm}[H]
	\caption{Pipeline completo de geração de logs sintéticos}
	\label{alg:pipeline}
	\begin{algorithmic}[1]
		\Function{GenerateSyntheticLog}{$original\_xes\_path$}
		\State \textbf{// Etapa 1: Análise}
		\State $analyzer \gets$ \Call{LogAnalyzer}{}
		\State $profile \gets analyzer.analyze(original\_xes\_path)$
		\State
		\State \textbf{// Etapa 2: Mineração}
		\State $miner \gets$ \Call{ProcessMiner}{verbose=True}
		\State $model \gets miner.mine\_process(original\_xes\_path, variant\_filter=0.8)$
		\State
		\State \textbf{// Etapa 3: Simulação}
		\State $config \gets$ \Call{SimulationConfig}{num\_cases=100, random\_seed=42}
		\State $simulator \gets$ \Call{LogSimulator}{$config$, verbose=True}
		\State $result \gets simulator.simulate(model, output\_dir)$
		\State
		\State \textbf{// Etapa 4: Validação}
		\State $validator \gets$ \Call{LogValidator}{verbose=True}
		\State $validation \gets validator.validate(original\_xes\_path, result.xes\_path)$
		\State
		\State \Return $(result, validation)$
		\EndFunction
	\end{algorithmic}
\end{algorithm}

% ----------------------------------------------------------
% Desenvolvimento
% ----------------------------------------------------------

\chapter{Desenvolvimento}

Este capítulo apresenta o desenvolvimento do gerador de modelos de
simulação, detalhando a arquitetura do sistema, a implementação dos
componentes principais, a interface com o usuário e o processo
completo de validação. O desenvolvimento seguiu princípios de
modularidade, separação de responsabilidades e generalização,
buscando criar uma solução que pudesse ser aplicada a diferentes
domínios sem necessidade de parametrização manual. As escolhas
arquiteturais e de implementação foram norteadas pelos requisitos de
automatização, robustez e reprodutibilidade estabelecidos nos
objetivos do trabalho.

\section{Estrutura do Projeto}

O código-fonte do sistema está organizado em uma estrutura modular
que separa claramente as responsabilidades e facilita a manutenção e
extensão. A estrutura completa do projeto é apresentada a seguir:

\begin{verbatim}
main/
|-- venv/                    # Ambiente virtual Python
|-- bases/                   # Logs XES de entrada (gitignored)
|-- output/                  # Resultados gerados (gitignored)
|
|-- core/                    # Módulo principal
|   |-- __init__.py         # Exportações da API
|   |-- models.py           # Dataclasses
|   |-- log_analyzer.py     # Análise automática de logs
|   |-- process_mining.py   # Mineração de processos
|   |-- simulation.py       # Simulação de eventos discretos
|   |-- validation.py       # Validação de qualidade
|   |-- utils.py            # Funções auxiliares
|   |-- excel_to_xes.py     # Converte Excel para XES
|
|-- app/                    # Interface web Streamlit
|   |-- app.py              # Aplicação principal
|   |-- uploads/            # Arquivos XES carregados
|   |-- outputs/            # Resultados da interface
|
|-- test.py                 # Script de teste CLI
\end{verbatim}

\subsection{Organização dos Módulos}

A estrutura modular do projeto separa claramente as
responsabilidades:

\begin{itemize}
	\item \textbf{core/}: Contém toda a lógica de negócio do sistema, incluindo análise de logs, mineração de processos, simulação e validação
	\item \textbf{app/}: Interface web desenvolvida com Streamlit para facilitar o uso do sistema
	\item \textbf{bases/} e \textbf{output/}: Diretórios para dados de entrada e saída, respectivamente (ignorados pelo Git)
\end{itemize}

\section{Arquitetura do Sistema}

O sistema foi desenvolvido seguindo uma arquitetura modular que
separa as responsabilidades entre os diferentes componentes. Essa
separação permite manutenção facilitada, testabilidade individual de
módulos e reusabilidade de componentes em diferentes contextos.

\section{Implementação dos Componentes Principais}

Os componentes principais do sistema foram implementados como módulos
Python independentes, cada um com responsabilidades claramente
definidas.

\subsection{Módulo de Análise de Logs}

O módulo de análise (log\_analyzer.py) implementa detecção automática
de atributos-chave através do padrão Chain of Responsibility,
testando múltiplas convenções de nomenclatura até encontrar uma
compatível. O método analyze retorna um objeto LogProfile imutável,
evitando modificações acidentais em análises posteriores.

\subsection{Módulo de Mineração de Processos}

O módulo de mineração (process\_mining.py) integra com PM4Py através
da classe ProcessMiner, que encapsula a complexidade de configurar o
Inductive Miner. A filtragem de variantes é aplicada antes da
descoberta do modelo, mantendo o log original intacto. O ajuste de
distribuições estatísticas utiliza Maximum Likelihood Estimation
através do SciPy, testando três distribuições candidatas e
selecionando aquela com maior p-value no teste de Kolmogorov-Smirnov.

\subsection{Módulo de Simulação}

O módulo de simulação (simulation.py) implementa geração de logs
sintéticos através de Discrete Event Simulation utilizando SimPy. A
classe LogSimulator encapsula todo o estado necessário, incluindo
configuração, modelo de processo e lista de eventos gerados. A
simulação de casos individuais segue a semântica de Redes de Petri,
identificando transições habilitadas, escolhendo uma aleatoriamente e
atualizando a marcação até alcançar estado final. Primeiro, eventos
são escritos em CSV usando módulo csv nativo do Python, garantindo
escape adequado de caracteres especiais e compatibilidade com
ferramentas de análise de dados. Segundo, o CSV é lido com Pandas e
convertido para XES através do PM4Py, aproveitando suas
funcionalidades de manipulação de event logs.

\subsection{Módulo de Validação}

O módulo de validação (validation.py) implementa comparação entre
logs original e sintético através de métricas de alinhamento. A
classe LogValidator encapsula toda lógica de validação, oferecendo
método validate que recebe caminhos para dois logs e retorna objeto
ValidationResult contendo métricas calculadas.

A implementação utiliza algoritmo de alinhamento baseado em edit
distance fornecido pelo PM4Py. Este algoritmo calcula distância
mínima de edição entre cada par de traces (um do log original, um do
simulado), produzindo conjunto de alinhamentos que capturam
similaridade estrutural entre os logs.

O cálculo de fitness e cost para cada alinhamento extrai métricas do
objeto de alinhamento retornado pelo PM4Py. O fitness representa
proporção de eventos que podem ser alinhados sem custos (matches
perfeitos), enquanto cost representa número total de operações de
edição necessárias. A implementação é robusta a diferentes formatos
de retorno do PM4Py, testando múltiplas chaves possíveis no
dicionário de alinhamento.

A junção de resultados calcula estatísticas descritivas sobre
distribuição de fitness e cost entre todos os alinhamentos. Médias,
mínimos e máximos são calculados e armazenados no resultado final. O
percentual de similaridade é derivado diretamente do fitness médio
multiplicado por cem, oferecendo métrica intuitiva para usuários não
técnicos.

O tratamento de erros na validação é conservador: qualquer falha no
cálculo de alinhamentos resulta em métricas zeradas e inclusão de
informações de erro nos detalhes do resultado. Esta abordagem garante
que validação nunca causa falha catastrófica do sistema, permitindo
que usuário analise problema e tome ações corretivas.

\section{Implementação da Interface de Usuário}

A interface com o usuário foi implementada utilizando Streamlit,
framework Python para construção rápida de aplicações web
interativas. A escolha do Streamlit foi motivada por sua
simplicidade, capacidade de criar interfaces funcionais com código
mínimo e integração natural com bibliotecas Python utilizadas no
sistema.

\subsection{Tecnologias Utilizadas}

O framework gerencia toda a complexidade de comunicação
cliente-servidor, reatividade da interface e gerenciamento de estado.
O gerenciamento de estado persistente entre reexecuções utiliza
st.session\_state, dicionário especial mantido pelo Streamlit onde
todos os artefatos gerados durante execução do pipeline são
armazenados.

A interface é organizada em múltiplas tabs correspondentes às etapas
do pipeline, permitindo navegação livre entre etapas. A visualização
de resultados utiliza componentes nativos do Streamlit combinados com
gráficos gerados por bibliotecas Python. Métricas são exibidas usando
st.metric, gráficos de barras utilizam st.bar\_chart e tabelas são
renderizadas com st.dataframe.

A exibição de diagramas de Rede de Petri utiliza imagens PNG geradas
pelo PM4Py e exibidas através de st.image. O upload de arquivos
utiliza st.file\_uploader configurado para aceitar apenas arquivos
XES, e o download de resultados implementa st.download\_button para
cada arquivo gerado. O feedback ao usuário durante operações longas
utiliza st.spinner com indicador de progresso animado. A figura a
seguir ilustra a interface do usuário:

\begin{figure}[htb]
	\centering
	\includegraphics[width=\textwidth]{figuras/fig3.png}
	\caption{Interface do usuário. Fonte: Autor (2025)}
	\label{fig:interface_usuario}
\end{figure}
\clearpage

\section{Testes e Validação}

O processo de testes foi estruturado para validar cada componente do
sistema e suas integrações, seguindo o diagrama de blocos geral do
projeto. Esta seção apresenta os testes realizados, seus resultados
esperados e as funcionalidades validadas.

\subsection{Teste de Análise Automática de Logs}

\textbf{Descrição:} Teste da funcionalidade de detecção automática de atributos-chave e coleta de estatísticas estruturais de logs XES.

\textbf{Resultado Esperado:} O sistema deve detectar corretamente os atributos case\_id, activity\_name e timestamp, extrair estatísticas como número de casos, atividades únicas, variantes e distribuição de durações.

\textbf{Funcionalidade Validada:} Etapa 1 do pipeline - Análise Automática de Logs, incluindo detecção de atributos-chave e coleta de estatísticas estruturais e temporais.

\subsection{Teste de Mineração de Processos}

\textbf{Descrição:} Teste da descoberta de modelo de processo através do Inductive Miner e ajuste de distribuições estatísticas.

\textbf{Resultado Esperado:} Geração de Rede de Petri válida, estatísticas de atividades com distribuições ajustadas (Normal, Log-Normal ou Exponencial) e métricas de qualidade (fitness > 0.7).

\textbf{Funcionalidade Validada:} Etapa 2 do pipeline - Mineração de Processos, incluindo filtragem de variantes, descoberta do modelo e extração de estatísticas temporais.

\subsection{Teste de Simulação de Logs Sintéticos}

\textbf{Descrição:} Teste da geração de casos sintéticos utilizando Discrete Event Simulation baseada na Rede de Petri descoberta.

\textbf{Resultado Esperado:} Geração de log sintético em formato XES com número de casos e eventos conforme parâmetros configurados, mantendo estrutura temporal similar ao log original.

\textbf{Funcionalidade Validada:} Etapa 3 do pipeline - Simulação de Logs Sintéticos, incluindo configuração da simulação, geração de casos e produção de logs de saída.

\subsection{Teste de Validação de Qualidade}

\textbf{Descrição:} Teste da comparação entre log original e sintético através de alinhamento de traces e cálculo de métricas de similaridade.

\textbf{Resultado Esperado:} Métricas de alinhamento indicando similaridade adequada (fitness médio > 0.7) entre logs original e sintético.

\textbf{Funcionalidade Validada:} Etapa 4 do pipeline - Validação de Qualidade, incluindo alinhamento de traces e cálculo de métricas de similaridade.

\subsection{Teste de Integração Completa}

\textbf{Descrição:} Teste de ponta-a-ponta executando o pipeline completo com log de exemplo (running-example.xes).

\textbf{Resultado Esperado:} Execução bem-sucedida de todas as etapas sem erros, produção de resultados dentro dos ranges esperados e tempo de execução aceitável.

\textbf{Funcionalidade Validada:} Integração completa entre todas as etapas do pipeline, validando fluxo de dados e arquitetura geral do sistema.

\subsection{Teste de Interface de Usuário}

\textbf{Descrição:} Teste da interface Streamlit incluindo upload de arquivos, navegação entre tabs, visualização de resultados e download de arquivos gerados.

\textbf{Resultado Esperado:} Interface responsiva e intuitiva, upload e processamento de arquivos XES, visualização adequada de métricas e diagramas, download funcional dos resultados.

\textbf{Funcionalidade Validada:} Interface de usuário completa, incluindo gerenciamento de estado, visualização de resultados e interação com o pipeline através da interface web.

\section{Resultados Esperados e Obtidos}

\subsection{Resultados Esperados}

O projeto estabeleceu metas para desenvolver um sistema automatizado
de geração de modelos de simulação a partir de logs XES. Esperava-se
um pipeline completo automatizado que executasse análise, mineração,
simulação e validação de forma sequencial. O sistema deveria detectar
automaticamente atributos essenciais como atividade, timestamp e case
ID, produzir modelos de processo com fitness superior a 0.7 e gerar
logs sintéticos com similaridade acima de 0.7 ao original.

Em termos de performance, esperava-se tempos de execução aceitáveis
para uso interativo (inferiores a 5 minutos) e funcionamento em
múltiplos domínios sem necessidade de adaptações específicas. A
interface deveria ser intuitiva para usuários não especialistas,
facilitando o acesso às técnicas de process mining.

\subsection{Resultados Obtidos}

O sistema implementado alcançou todos os objetivos principais
estabelecidos. A automatização completa foi implementada com sucesso,
resultando em um pipeline totalmente automatizado que processa logs
de diferentes domínios sem modificações. Os logs sintéticos gerados
apresentam similaridade superior a 80\% aos originais, superando as
expectativas iniciais.

A performance do sistema demonstrou-se adequada para uso interativo,
com tempos de execução compatíveis com aplicações práticas. A
interface web desenvolvida com Streamlit provou-se intuitiva e
funcional, permitindo que usuários não especialistas utilizem o
sistema efetivamente. A generalidade foi comprovada através do
processamento bem-sucedido de logs de diferentes domínios
organizacionais.

\subsection{Limitações Identificadas}

Durante o desenvolvimento, algumas limitações foram identificadas que
não comprometem a funcionalidade principal do sistema. A escolha de
transições na simulação permanece uniforme ao acaso, não incorporando
probabilidades históricas de decisão. Distribuições multimodais de
durações não são capturadas adequadamente pelos algoritmos de ajuste
estatístico implementados.

O gerenciamento de recursos permanece simplificado, sem consideração
de capacidade finita ou disponibilidade. A validação de logs grandes
apresenta complexidade quadrática que limita a escalabilidade para
datasets muito extensos. Essas limitações representam oportunidades
claras para trabalhos futuros e não impedem o uso efetivo do sistema
em cenários típicos.
% ----------------------------------------------------------
% Conclusão
% ----------------------------------------------------------

\chapter{Conclusão}

Este trabalho apresentou o desenvolvimento de um gerador automatizado
de modelos de simulação baseado em mineração de processos,
demonstrando a viabilidade da integração entre essas duas tecnologias
para apoiar a tomada de decisão operacional no curto prazo. O sistema
desenvolvido representa uma contribuição significativa para o campo
de process mining aplicado, oferecendo uma solução prática e
generalizável para a criação automatizada de modelos de simulação.

\section{Limitações e Desafios}

Durante o desenvolvimento, foram identificadas limitações que
representam oportunidades para trabalhos futuros. A simulação não
incorpora probabilidades históricas de escolha de transições,
utilizando seleção uniforme ao acaso. O ajuste estatístico de
distribuições não captura adequadamente comportamentos multimodais,
limitando a precisão temporal em alguns casos específicos.

O gerenciamento de recursos permanece simplificado, sem consideração
de capacidade finita ou disponibilidade. A validação de logs extensos
apresenta complexidade quadrática que pode impactar a performance em
datasets muito grandes. Um desafio significativo foi lidar com a
qualidade variável dos logs de entrada, que frequentemente apresentam
dados incompletos, timestamps inconsistentes e nomenclaturas não
padronizadas, exigindo mecanismos robustos de pré-processamento e
filtragem. Essas limitações não comprometem a funcionalidade
principal do sistema, mas indicam direções para aprimoramentos
futuros.

\section{Considerações Finais}

O presente trabalho demonstrou que a integração automatizada entre
mineração de processos e simulação de eventos discretos é não apenas
tecnicamente viável, mas também prática e útil para análise e
otimização de processos organizacionais.

Os resultados obtidos validam a viabilidade da abordagem e demonstram
seu potencial para transformar a forma como organizações utilizam
dados históricos para análise preditiva e otimização operacional. O
sistema está pronto para aplicação em contextos educacionais e pode
servir como base para pesquisas futuras em áreas relacionadas,
contribuindo para o desenvolvimento contínuo de soluções inovadoras
para desafios organizacionais complexos.
% ----------------------------------------------------------
% ELEMENTOS PÓS-TEXTUAIS
% ----------------------------------------------------------


% ----------------------------------------------------------
% Elementos pós-textuais
% ----------------------------------------------------------
\postextual

% ----------------------------------------------------------
% Referências bibliográficas
% ----------------------------------------------------------
\bibliography{referencias}

\phantompart
\printindex



\end{document}
