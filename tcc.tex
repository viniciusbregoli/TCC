\documentclass[
    12pt,               % Tamanho da fonte
    openright,          % Capítulos começam em página ímpar
    oneside,            % Impressão em um lado
    a4paper,            % Tamanho do papel
    brazil              % Idioma principal
]{abntex2}

% ---
% Pacotes básicos 
% ---
\usepackage{times}             % Usa fonte Times New Roman
\usepackage[T1]{fontenc}        % Seleção de códigos de fonte
\usepackage[utf8]{inputenc}     % Codificação do documento
\usepackage{indentfirst}        % Indenta o primeiro parágrafo
\usepackage{color}              % Controle das cores
\usepackage{graphicx}           % Inclusão de gráficos
\usepackage{microtype}          % Para melhorias de justificação
\usepackage{lastpage}           % Para referência à última página

% ---
% Pacotes de citações
% ---
\usepackage[brazilian,hyperpageref]{backref}     % Páginas com as citações
\usepackage[alf]{abntex2cite}   % Citações padrão ABNT

% ---
% Configurações do pacote backref
% ---
\renewcommand{\backrefpagesname}{Citado na(s) página(s):~}
\renewcommand{\backref}{}
\renewcommand*{\backrefalt}[4]{
    \ifcase #1 %
        Nenhuma citação no texto.%
    \or
        Citado na página #2.%
    \else
        Citado #1 vezes nas páginas #2.%
    \fi}%

% ---
% Informações do documento
% ---
\titulo{DESENVOLVIMENTO DE GERADOR DE MODELOS DE SIMULAÇÃO PARA TOMADA DE DECISÃO NO CURTO PRAZO USANDO PROCESS MINING}
\autor{VINÍCIUS LEOBET BREGOLI}
\local{Curitiba}
\data{2025}
\orientador{Prof. Dr. Edson Emílio Scalabrin}
\instituicao{%
  PONTIFÍCIA UNIVERSIDADE CATÓLICA DO PARANÁ -- PUCPR
  \par
  ESCOLA POLITÉCNICA
  \par
  CURSO DE ENGENHARIA DE COMPUTAÇÃO}
\tipotrabalho{Trabalho de Conclusão de Curso}
\preambulo{Trabalho de Conclusão de Curso apresentado ao Curso de Engenharia de Computação da Pontifícia Universidade Católica do Paraná como requisito parcial para obtenção do grau de Bacharel em Engenharia de Computação.}

% ---
% Configurações de aparência do PDF
% ---
\definecolor{blue}{RGB}{41,5,195}

\makeatletter
\hypersetup{
    pdftitle={\@title},
    pdfauthor={\@author},
    pdfsubject={\imprimirpreambulo},
    pdfcreator={LaTeX with abnTeX2},
    pdfkeywords={abnt}{latex}{tcc}{engenharia de computação},
    colorlinks=true,
    linkcolor=blue,
    citecolor=blue,
    filecolor=magenta,
    urlcolor=blue,
    bookmarksdepth=4,
    pdfstartview=FitH,
    unicode=true
}
\makeatother

% Corrigir warnings do hyperref para títulos com acentos
\pdfstringdefDisableCommands{%
  \def\\{}%
  \def\textbf#1{<#1>}%
  \def\textit#1{<#1>}%
  \def\uppercase#1{#1}%
}

% ---
% Configurações de formatação ABNT
% ---

% Margens: 3cm superior e esquerda, 2cm inferior e direita
\usepackage[
    top=3cm,
    left=3cm,
    right=2cm,
    bottom=2cm
]{geometry}

% Espaçamento 1,5 no texto
\usepackage{setspace}

% Configurações de parágrafo
\setlength{\parindent}{1.3cm}
\setlength{\parskip}{0.2cm}

% Justificação do texto (já é padrão no LaTeX, mas garantindo)
\usepackage{ragged2e}
\justifying

% Garantir que títulos também usem Times New Roman (ABNT exige mesma fonte)
\renewcommand{\ABNTEXchapterfont}{\rmfamily\bfseries}
\renewcommand{\ABNTEXsectionfont}{\rmfamily\bfseries}
\renewcommand{\ABNTEXsubsectionfont}{\rmfamily\bfseries}
\renewcommand{\ABNTEXsubsubsectionfont}{\rmfamily\bfseries}

% Configuração para citações longas (espaçamento simples)
\renewenvironment{citacao}{%
    \small
    \begin{singlespace}
    \begin{list}{}{%
        \setlength{\leftmargin}{4cm}%
        \setlength{\parsep}{0pt}%
        \setlength{\parskip}{0pt}%
        \setlength{\itemsep}{0pt}%
    }
    \item\relax
}{%
    \end{list}
    \end{singlespace}
}

% ---
% Compila o índice
% ---
\makeindex

% ----
% Início do documento
% ----
\begin{document}

\selectlanguage{brazil}
\frenchspacing

% ----------------------------------------------------------
% ELEMENTOS PRÉ-TEXTUAIS
% ----------------------------------------------------------

% ---
% Capa
% ---
\imprimircapa

% ---
% Folha de rosto
% ---
\imprimirfolhaderosto*

% ---
% RESUMOS
% ---

% Resumo em português
\setlength{\absparsep}{18pt}
\begin{resumo}

	A tomada de decisão no curto prazo em ambientes complexos, como
	centros cirúrgicos, exige ferramentas capazes de integrar dados
	históricos e informações em tempo real. Nesse contexto, a mineração
	de processos (Process Mining - PM) e a simulação computacional
	emergem como tecnologias para apoiar gestores na alocação eficiente
	de recursos, detecção de desvios e previsão de cenários. Este
	trabalho propõe o desenvolvimento de um gerador de modelos de
	simulação baseado em PM, com foco em reduzir o esforço humano e o
	tempo necessário para a construção de modelos. O gerador utiliza logs
	de eventos como fonte de dados para criar automaticamente modelos de
	simulação, permitindo avaliar alternativas de curto prazo e apoiar a
	tomada de decisão operacional. O estudo se apoia no framework PM4SOS,
	estendido e adaptado para o domínio hospitalar, e integra métodos
	multicritério e técnicas de otimização. O resultado esperado é a
	disponibilização de um protótipo que auxilie gestores a analisar
	filas, prever ocupação de salas, otimizar agendas e reduzir gargalos,
	contribuindo para maior eficiência operacional, redução de custos e
	melhor qualidade no atendimento. \vspace{\onelineskip}

	\noindent
	\textbf{Palavras-chave}: Mineração
	de Processos, Simulação Computacional, Tomada de Decisão, Otimização,
	Agendamento.
\end{resumo}

% Resumo em inglês
\begin{resumo}[Abstract]
	\begin{otherlanguage*}{english}
		Short-term decision-making in complex environments, such as surgical centers, requires tools capable of integrating historical data and real-time information. In this context, Process Mining (PM) and computer simulation emerge as key technologies to support managers in efficient resource allocation, deviation detection, and scenario prediction. This work proposes the development of a simulation model generator based on PM, focusing on reducing human effort and the time required to build models. The generator uses event logs as a data source to automatically create simulation models, allowing the evaluation of short-term alternatives and supporting operational decision-making. The study is based on the PM4SOS framework, extended and adapted to the hospital domain, and integrates multicriteria methods and optimization techniques. The expected outcome is a prototype that helps managers analyze queues, predict room occupancy, optimize schedules, and reduce bottlenecks, contributing to greater operational efficiency, cost reduction, and improved service quality.
		\vspace{\onelineskip}

		\noindent
		\textbf{Keywords}: Process Mining, Computer Simulation, Decision-Making, Optimization, Scheduling.
	\end{otherlanguage*}
\end{resumo}

% ---
% Lista de ilustrações
% ---
\pdfbookmark[0]{\listfigurename}{lof}
\listoffigures*
\cleardoublepage

% ---
% Lista de tabelas
% ---
\pdfbookmark[0]{\listtablename}{lot}
\listoftables*
\cleardoublepage

% ---
% Sumário
% ---
\pdfbookmark[0]{\contentsname}{toc}
\tableofcontents*
\cleardoublepage

% ----------------------------------------------------------
% ELEMENTOS TEXTUAIS
% ----------------------------------------------------------
\textual

% ----------------------------------------------------------
% Introdução
% ----------------------------------------------------------
\chapter{Introdução}

Em organizações modernas, a complexidade operacional e o dinamismo
dos processos exigem decisões cada vez mais rápidas e embasadas em
dados. Setores como saúde, manufatura, logística e serviços
compartilham desafios semelhantes: alocação eficiente de recursos,
detecção de gargalos, redução de custos e melhoria contínua de
desempenho. Em todos esses contextos, as decisões de curto prazo —
aquelas que precisam ser tomadas em horas ou dias — têm impacto
direto na produtividade e na qualidade do serviço prestado.

Apesar do avanço dos sistemas de informação e do grande volume de
dados disponíveis, a transformação desses dados em conhecimento útil
para a tomada de decisão ainda depende, em grande parte, de análise
manual e da experiência de gestores e especialistas. Esse processo,
além de demorado, está sujeito a erros humanos e vieses cognitivos,
dificultando a reação rápida diante de mudanças operacionais.

A mineração de processos (Process Mining – PM) surge como uma
abordagem capaz de extrair, a partir de logs de eventos, informações
estruturadas sobre o comportamento real dos processos. Essa técnica
permite descobrir modelos de processos, analisar desvios e
identificar gargalos com base em dados reais de execução. No entanto,
por si só, a mineração de processos não oferece mecanismos preditivos
para antecipar cenários futuros ou testar alternativas operacionais.

Por outro lado, a simulação computacional é amplamente utilizada para
avaliar o impacto de decisões antes de sua aplicação prática. Modelos
de simulação permitem experimentar diferentes cenários, mensurar
indicadores de desempenho e prever resultados. Contudo, a construção
manual desses modelos é uma tarefa complexa, que exige tempo,
conhecimento técnico e compreensão detalhada do processo — fatores
que limitam sua aplicação em situações que demandam respostas
rápidas.

Dessa forma, o problema central abordado neste trabalho consiste em
como automatizar a criação de modelos de simulação baseados em dados
reais, extraídos por mineração de processos, para apoiar a tomada de
decisão operacional em curto prazo.

Embora a tese de \cite{ferronato2022} tenha proposto o framework
PM4SOS para o contexto hospitalar, voltado ao agendamento cirúrgico,
a aplicação desse conceito pode ser expandida para outros domínios,
mantendo a mesma lógica de integração entre mineração de processos,
simulação e otimização. Este trabalho propõe, portanto, generalizar a
abordagem, tornando o gerador de modelos aplicável a qualquer
ambiente que possua processos registrados em sistemas de informação,
tendo o setor hospitalar como estudo de caso para validação prática.

A pesquisa busca preencher lacunas existentes na literatura e na
prática organizacional, especialmente no que diz respeito à
integração automatizada entre dados históricos e modelos de
simulação, reduzindo o esforço cognitivo do tomador de decisão e
possibilitando análises preditivas em tempo reduzido.

\section{Contexto e Problema}

Em organizações modernas, a complexidade operacional e o dinamismo
dos processos exigem decisões cada vez mais rápidas e embasadas em
dados. Setores como saúde, manufatura, logística e serviços
compartilham desafios semelhantes: alocação eficiente de recursos,
detecção de gargalos, redução de custos e melhoria contínua de
desempenho. Em todos esses contextos, as decisões de curto prazo —
aquelas que precisam ser tomadas em horas ou dias — têm impacto
direto na produtividade e na qualidade do serviço prestado.

Apesar do avanço dos sistemas de informação e do grande volume de
dados disponíveis, a transformação desses dados em conhecimento útil
para a tomada de decisão ainda depende, em grande parte, de análise
manual e da experiência de gestores e especialistas. Esse processo,
além de demorado, está sujeito a erros humanos e vieses cognitivos,
dificultando a reação rápida diante de mudanças operacionais.

A mineração de processos surge como uma abordagem capaz de extrair, a
partir de logs de eventos, informações estruturadas sobre o
comportamento real dos processos. Essa técnica permite descobrir
modelos de processos, analisar desvios e identificar gargalos com
base em dados reais de execução. No entanto, por si só, a mineração
de processos não oferece mecanismos preditivos para antecipar
cenários futuros ou testar alternativas operacionais.

Por outro lado, a simulação computacional é amplamente utilizada para
avaliar o impacto de decisões antes de sua aplicação prática. Modelos
de simulação permitem experimentar diferentes cenários, mensurar
indicadores de desempenho e prever resultados. Contudo, a construção
manual desses modelos é uma tarefa complexa, que exige tempo,
conhecimento técnico e compreensão detalhada do processo — fatores
que limitam sua aplicação em situações que demandam respostas
rápidas.

Dessa forma, o problema central abordado neste trabalho consiste em
como automatizar a criação de modelos de simulação baseados em dados
reais, extraídos por mineração de processos, para apoiar a tomada de
decisão operacional em curto prazo.

Embora a tese de Ferronato (2022) tenha proposto o framework PM4SOS
para o contexto hospitalar, voltado ao agendamento cirúrgico, a
aplicação desse conceito pode ser expandida para outros domínios,
mantendo a mesma lógica de integração entre mineração de processos,
simulação e otimização. Este trabalho propõe, portanto, generalizar a
abordagem, tornando o gerador de modelos aplicável a qualquer
ambiente que possua processos registrados em sistemas de informação,
tendo o setor hospitalar como estudo de caso para validação prática.

A pesquisa busca preencher lacunas existentes na literatura e na
prática organizacional, especialmente no que diz respeito à
integração automatizada entre dados históricos e modelos de
simulação, reduzindo o esforço cognitivo do tomador de decisão e
possibilitando análises preditivas em tempo reduzido.

\section{Motivação}

A crescente digitalização dos processos organizacionais tem gerado um
volume significativo de dados operacionais, os quais representam uma
fonte valiosa de informações para análise e melhoria contínua. No
entanto, a maior parte desses dados permanece subutilizada, sendo
empregada apenas de forma descritiva, sem oferecer suporte efetivo à
tomada de decisão operacional em tempo hábil. Esse cenário evidencia
a necessidade de ferramentas capazes de transformar registros
históricos e dados em tempo real em conhecimento acionável para
apoiar decisões rápidas e precisas.

A mineração de processos se apresenta como uma tecnologia capaz de
extrair conhecimento a partir de logs de eventos, descrevendo o
comportamento real dos processos. Já a simulação computacional
complementa essa análise ao permitir a avaliação de cenários
alternativos, fornecendo uma visão preditiva e exploratória do
sistema em estudo. Entretanto, a criação manual de modelos de
simulação ainda representa um gargalo, exigindo alto esforço
cognitivo e conhecimento técnico especializado.

A motivação principal deste trabalho reside na necessidade de
automatizar a geração de modelos de simulação utilizando informações
obtidas por meio da mineração de processos. Essa automatização tem
potencial para reduzir significativamente o tempo e o esforço
necessários para construção de modelos analíticos, além de ampliar o
acesso de gestores e analistas a ferramentas de apoio à decisão em
ambientes de alta complexidade.

Este trabalho propõe criar um gerador de modelos de simulação de
forma generalizável a diferentes contextos organizacionais, mantendo
o ambiente hospitalar apenas como estudo de caso para validação
experimental.

Ao promover a integração entre mineração de processos, simulação e
otimização multicritério, este estudo busca contribuir para o avanço
das práticas de gestão operacional baseada em dados, oferecendo uma
alternativa viável e automatizada para apoio à decisão no curto
prazo.

\section{Estado da Arte}

O estado atual das tecnologias de simulação e modelagem para tomada
de decisão apresenta diversas abordagens e metodologias. No contexto
acadêmico, as principais áreas de pesquisa incluem:

\subsection{Modelos de Simulação Discreta}

Os modelos de simulação por eventos discretos têm sido amplamente
utilizados para modelar processos complexos em ambientes
educacionais. Estes modelos permitem a representação de sistemas
dinâmicos onde mudanças de estado ocorrem em pontos específicos no
tempo.

\subsection{Sistemas de Apoio à Decisão (SAD)}

Os Sistemas de Apoio à Decisão combinam modelos analíticos, técnicas
de simulação e interfaces de usuário para auxiliar gestores na tomada
de decisões complexas. No contexto educacional, estes sistemas têm
sido aplicados em áreas como:

\begin{itemize}
	\item Planejamento de horários e alocação de salas
	\item Gestão de recursos humanos
	\item Análise de desempenho acadêmico
	\item Otimização orçamentária
\end{itemize}

\subsection{Técnicas de Project Management}

As metodologias de gerenciamento de projetos, como PMI (Project
Management Institute) e PRINCE2, fornecem frameworks estruturados
para planejamento, execução e controle de projetos. A aplicação
dessas técnicas no desenvolvimento de sistemas de simulação garante
maior eficiência e qualidade no processo de desenvolvimento.

\section{Soluções Similares}

Uma análise das soluções existentes no mercado revela diferentes
abordagens para problemas similares:

\subsection{Sistemas Comerciais}

\begin{itemize}
	\item \textbf{Arena Simulation Software}: Ferramenta robusta para modelagem e simulação de processos, amplamente utilizada na indústria
	\item \textbf{AnyLogic}: Plataforma de simulação multimethod que combina diferentes paradigmas de modelagem
	\item \textbf{MATLAB Simulink}: Ambiente de simulação para sistemas dinâmicos com forte base matemática
\end{itemize}

\subsection{Soluções Acadêmicas}

\begin{itemize}
	\item \textbf{NetLogo}: Ambiente de modelagem baseado em agentes, popular em pesquisas acadêmicas
	\item \textbf{R e Python}: Linguagens de programação com extensas bibliotecas para simulação e análise estatística
	\item \textbf{Gephi}: Ferramenta de visualização e análise de redes complexas
\end{itemize}

\subsection{Limitações das Soluções Existentes}

Apesar da variedade de ferramentas disponíveis, observa-se algumas
limitações:

\begin{itemize}
	\item Falta de especialização para o contexto acadêmico específico
	\item Complexidade excessiva para usuários não técnicos
	\item Custos elevados de licenciamento
	\item Dificuldade de integração com sistemas acadêmicos existentes
	\item Ausência de templates específicos para cenários educacionais
\end{itemize}

\section{Objetivos}

\subsection{Objetivo Geral}

Desenvolver um \textbf{gerador de modelos de simulação baseado em
	mineração de processos} para apoiar a \textbf{tomada de decisão no
	curto prazo}, capaz de criar automaticamente modelos de simulação a
partir de logs de eventos, reduzindo o esforço humano e o tempo
necessário para a modelagem de sistemas complexos.

\subsection{Objetivos Específicos}

Para alcançar o objetivo geral, este trabalho busca atender aos
seguintes objetivos específicos:

\begin{itemize}
	\item Investigar métodos e técnicas de integração entre mineração de
	      processos, simulação computacional e otimização multicritério;
	\item Projetar uma arquitetura de sistema capaz de gerar automaticamente
	      modelos de simulação a partir de logs de eventos processados por
	      ferramentas de PM;
	\item Implementar um protótipo funcional do gerador de modelos, com base em
	      bibliotecas de mineração de processos e simulação (como PM4PY e
	      SimPy);
	\item Aplicar o protótipo desenvolvido em um estudo de caso no contexto
	      hospitalar, validando sua eficácia na geração de modelos e apoio à
	      decisão operacional;
	\item Avaliar o desempenho do sistema proposto quanto à precisão dos
	      modelos gerados, tempo de execução e potencial de generalização para
	      outros domínios.
\end{itemize}

\section{Justificativa}

A crescente complexidade dos processos operacionais nas organizações
exige soluções capazes de transformar grandes volumes de dados em
informações relevantes para apoiar a tomada de decisão. Em muitos
cenários, os gestores precisam reagir rapidamente a mudanças de
demanda, restrições de recursos ou falhas inesperadas, o que torna
essencial o uso de ferramentas que combinem análise de dados,
previsão e simulação.

Nesse contexto, a mineração de processos (Process Mining – PM) e a
simulação computacional se destacam como abordagens complementares. A
primeira permite compreender o comportamento real dos processos a
partir de registros de eventos, enquanto a segunda possibilita
avaliar o impacto de decisões antes de sua implementação prática. A
integração entre essas técnicas viabiliza a criação de sistemas de
apoio à decisão mais precisos, adaptáveis e orientados por dados.

Entretanto, a construção manual de modelos de simulação ainda
representa uma barreira significativa para a aplicação ampla dessas
tecnologias, especialmente em contextos onde o tempo é um fator
crítico. A necessidade de profissionais especializados, aliada à
complexidade dos processos e à variabilidade dos dados, limita o uso
da simulação como ferramenta de suporte à decisão operacional em
tempo real.

Dessa forma, este trabalho se justifica pela necessidade de tornar a
geração de modelos de simulação mais automatizada e genérica,
reduzindo a dependência de modelagem manual e ampliando o uso desta
técnica para diferentes domínios. Ao propor um gerador de modelos de
simulação baseado em PM, busca-se oferecer uma ferramenta flexível,
capaz de apoiar decisões de curto prazo em ambientes complexos e
dinâmicos, promovendo ganhos de eficiência operacional, qualidade e
confiabilidade na gestão de processos.

% ----------------------------------------------------------
% Referencial Teórico
% ----------------------------------------------------------
\chapter{Referencial Teórico}

Este capítulo apresenta os conceitos fundamentais necessários para o
entendimento do trabalho desenvolvido, abordando as bases teóricas de
simulação, sistemas de apoio à decisão e metodologias de
gerenciamento de projetos.

\section{Fundamentos de Simulação}

A simulação é uma técnica poderosa para modelar e analisar sistemas
complexos, permitindo a experimentação com diferentes cenários sem os
custos e riscos associados à implementação real.

\subsection{Simulação por Eventos Discretos}

A simulação por eventos discretos (DES - Discrete Event Simulation) é
uma metodologia que modela sistemas onde mudanças de estado ocorrem
em pontos específicos no tempo. Este paradigma é particularmente
adequado para sistemas onde entidades (como estudantes, recursos ou
processos) interagem de forma discreta.

Características principais da DES:
\begin{itemize}
	\item Modelagem baseada em eventos que alteram o estado do sistema
	\item Controle temporal através de uma lista de eventos futuros
	\item Representação de entidades com atributos específicos
	\item Capacidade de modelar filas, recursos limitados e processos
	      estocásticos
\end{itemize}

\subsection{Modelos Matemáticos para Simulação}

Os modelos matemáticos fornecem a base quantitativa para simulações
precisas. No contexto acadêmico, estes modelos incluem:

\begin{itemize}
	\item \textbf{Modelos de Filas}: Para representar fluxos de estudantes e recursos
	\item \textbf{Modelos Estocásticos}: Para incorporar variabilidade e incerteza
	\item \textbf{Modelos de Otimização}: Para encontrar soluções ótimas dentro do espaço de simulação
	\item \textbf{Modelos Preditivos}: Para projetar cenários futuros baseados em dados históricos
\end{itemize}

\section{Sistemas de Apoio à Decisão}

Os Sistemas de Apoio à Decisão (SAD) combinam modelos analíticos,
bases de dados e interfaces de usuário para auxiliar gestores na
tomada de decisões complexas.

\subsection{Arquitetura de SAD}

Um SAD típico é composto por três componentes principais:

\begin{itemize}
	\item \textbf{Sistema de Gerenciamento de Dados}: Responsável pelo armazenamento e recuperação de informações
	\item \textbf{Sistema de Gerenciamento de Modelos}: Contém os algoritmos e modelos matemáticos
	\item \textbf{Sistema de Interface}: Facilita a interação entre usuário e sistema
\end{itemize}

\subsection{Aplicações em Ambientes Educacionais}

No contexto educacional, os SAD têm sido aplicados em diversas áreas:

\begin{itemize}
	\item Planejamento de horários e alocação de recursos
	\item Gestão orçamentária e financeira
	\item Análise de desempenho acadêmico e predição de resultados
	\item Otimização de processos administrativos
	\item Suporte ao planejamento estratégico institucional
\end{itemize}

\section{Project Management (PM)}

As metodologias de gerenciamento de projetos fornecem frameworks
estruturados para planejamento, execução e controle de projetos
complexos.

\subsection{Metodologias PM}

\subsubsection{PMI (Project Management Institute)}

O PMI define cinco grupos de processos e dez áreas de conhecimento
que formam a base do gerenciamento de projetos:

\textbf{Grupos de Processos:}
\begin{itemize}
	\item Iniciação
	\item Planejamento
	\item Execução
	\item Monitoramento e Controle
	\item Encerramento
\end{itemize}

\textbf{Áreas de Conhecimento Relevantes para este projeto:}
\begin{itemize}
	\item Gerenciamento de Escopo
	\item Gerenciamento de Cronograma
	\item Gerenciamento de Qualidade
	\item Gerenciamento de Riscos
\end{itemize}

\subsubsection{Metodologias Ágeis}

As metodologias ágeis complementam as práticas tradicionais de PM,
oferecendo flexibilidade e adaptabilidade:

\begin{itemize}
	\item \textbf{Scrum}: Framework para desenvolvimento iterativo
	\item \textbf{Kanban}: Sistema visual para gerenciamento de fluxo de trabalho
	\item \textbf{Lean}: Foco na eliminação de desperdícios e otimização de valor
\end{itemize}

\subsection{Aplicação de PM no Desenvolvimento de Sistemas}

A aplicação de técnicas de PM no desenvolvimento de sistemas de
simulação garante:

\begin{itemize}
	\item Controle rigoroso de escopo e requisitos
	\item Gerenciamento eficaz de cronograma e recursos
	\item Identificação e mitigação proativa de riscos
	\item Garantia de qualidade através de processos estruturados
	\item Comunicação eficiente entre stakeholders
\end{itemize}

\section{Tecnologias de Desenvolvimento}

\subsection{Linguagens e Frameworks para Simulação}

\subsubsection{Python}

Python é amplamente utilizado em simulação devido à sua sintaxe clara
e extenso ecossistema de bibliotecas:

\begin{itemize}
	\item \textbf{NumPy}: Computação numérica eficiente
	\item \textbf{SciPy}: Algoritmos científicos e estatísticos
	\item \textbf{SimPy}: Framework específico para simulação por eventos discretos
	\item \textbf{Matplotlib/Plotly}: Visualização de dados e resultados
\end{itemize}

\subsection{Tecnologias Web para Interface}

\subsubsection{Frontend Moderno}

Para interfaces de usuário responsivas e interativas:

\begin{itemize}
	\item \textbf{React.js}: Biblioteca para construção de interfaces reativas
	\item \textbf{TypeScript}: Superset do JavaScript com tipagem estática
	\item \textbf{D3.js}: Biblioteca para visualizações de dados interativas
\end{itemize}

\subsubsection{Backend e APIs}

Para comunicação entre interface e engine de simulação:

\begin{itemize}
	\item \textbf{Node.js}: Ambiente de execução JavaScript server-side
	\item \textbf{Express.js}: Framework web minimalista e flexível
	\item \textbf{REST APIs}: Arquitetura para comunicação cliente-servidor
\end{itemize}

% ----------------------------------------------------------
% Metodologia
% ----------------------------------------------------------
\chapter{Metodologia}

Este capítulo descreve a metodologia utilizada para o desenvolvimento
do gerador de modelos de simulação, aplicando técnicas de Project
Management para garantir a qualidade e eficiência do processo.

\section{Tipo de Pesquisa}

Esta pesquisa caracteriza-se como aplicada, com abordagem qualitativa
e quantitativa, utilizando métodos de desenvolvimento de software
especializado em simulação. A natureza aplicada da pesquisa visa
solucionar problemas práticos de tomada de decisão no ambiente
acadêmico através de ferramentas computacionais.

\section{Abordagem Metodológica}

O desenvolvimento seguiu uma metodologia híbrida que combina:

\begin{itemize}
	\item \textbf{Práticas de Project Management}: Baseadas no PMI para controle de escopo, cronograma e qualidade
	\item \textbf{Desenvolvimento Ágil}: Iterações curtas com feedback contínuo
	\item \textbf{Metodologia de Simulação}: Validação e verificação rigorosa dos modelos desenvolvidos
\end{itemize}

\section{Procedimentos Metodológicos}

O desenvolvimento seguiu uma abordagem estruturada em fases bem
definidas:

\begin{enumerate}
	\item \textbf{Análise de Requisitos Específicos para Simulação}
	      \begin{itemize}
		      \item Levantamento de cenários acadêmicos a serem modelados
		      \item Definição de parâmetros e variáveis de entrada
		      \item Especificação de métricas de saída e relatórios
	      \end{itemize}

	\item \textbf{Design da Arquitetura do Sistema}
	      \begin{itemize}
		      \item Separação entre engine de simulação e interface de usuário
		      \item Definição de APIs para comunicação entre componentes
		      \item Estruturação do banco de dados para armazenamento de modelos
	      \end{itemize}

	\item \textbf{Desenvolvimento do Engine de Simulação}
	      \begin{itemize}
		      \item Implementação dos algoritmos de simulação discreta
		      \item Desenvolvimento dos modelos matemáticos base
		      \item Criação do sistema de geração de relatórios
	      \end{itemize}

	\item \textbf{Desenvolvimento da Interface de Usuário}
	      \begin{itemize}
		      \item Interface drag-and-drop para criação de modelos
		      \item Componentes de visualização de resultados
		      \item Sistema de templates para cenários comuns
	      \end{itemize}

	\item \textbf{Integração e Testes Especializados}
	      \begin{itemize}
		      \item Testes de precisão dos algoritmos de simulação
		      \item Validação com cenários conhecidos
		      \item Testes de performance e escalabilidade
	      \end{itemize}

	\item \textbf{Validação e Documentação}
	      \begin{itemize}
		      \item Testes com usuários finais
		      \item Comparação com ferramentas existentes
		      \item Documentação técnica e de usuário
	      \end{itemize}
\end{enumerate}

\section{Ferramentas Utilizadas}

Para o desenvolvimento do sistema, foram utilizadas as seguintes
tecnologias:

\begin{itemize}
	\item \textbf{Frontend}: React.js, TypeScript, Material-UI
	\item \textbf{Backend}: Node.js, Express.js, PostgreSQL
	\item \textbf{Simulação}: Python com bibliotecas NumPy, SciPy e SimPy
	\item \textbf{Visualização}: D3.js, Chart.js
	\item \textbf{Ferramentas}: Git, Docker, Jest (testes), PyTest (testes Python)
\end{itemize}

\section{Plano de Testes}

O plano de testes foi estruturado para garantir a qualidade e
confiabilidade do sistema de simulação desenvolvido. Os testes foram
organizados em diferentes categorias para cobrir todos os aspectos
críticos do sistema.

\subsection{Testes Unitários}

Os testes unitários focam na validação de componentes individuais do
sistema:

\begin{itemize}
	\item \textbf{Algoritmos de Simulação}: Validação dos cálculos matemáticos e lógica de simulação
	\item \textbf{Funções de Entrada de Dados}: Verificação da validação e processamento de dados de entrada
	\item \textbf{Módulos de Geração de Relatórios}: Teste da corretude dos relatórios gerados
	\item \textbf{Componentes de Interface}: Validação do comportamento individual dos componentes UI
\end{itemize}

\subsection{Testes de Integração}

Os testes de integração verificam a comunicação entre diferentes
módulos:

\begin{itemize}
	\item \textbf{Frontend-Backend}: Comunicação via API REST
	\item \textbf{Backend-Engine de Simulação}: Integração entre a API e os algoritmos de simulação
	\item \textbf{Sistema de Persistência}: Integração com banco de dados
	\item \textbf{Módulos de Visualização}: Integração entre dados simulados e componentes gráficos
\end{itemize}

\subsection{Testes de Performance}

Os testes de performance avaliam o desempenho do sistema sob
diferentes cargas:

\begin{itemize}
	\item \textbf{Tempo de Execução}: Medição do tempo necessário para executar simulações de diferentes complexidades
	\item \textbf{Uso de Memória}: Monitoramento do consumo de memória durante simulações extensas
	\item \textbf{Escalabilidade}: Teste com múltiplas simulações simultâneas
	\item \textbf{Responsividade da Interface}: Tempo de resposta da interface durante operações intensivas
\end{itemize}

\subsection{Testes de Validação}

Os testes de validação verificam se os resultados das simulações são
consistentes e corretos:

\begin{itemize}
	\item \textbf{Validação Matemática}: Comparação com cálculos manuais para cenários simples
	\item \textbf{Benchmarking}: Comparação com ferramentas de simulação estabelecidas
	\item \textbf{Testes de Cenários Extremos}: Validação do comportamento em condições limite
	\item \textbf{Reprodutibilidade}: Verificação de que simulações com mesmos parâmetros produzem resultados consistentes
\end{itemize}

\subsection{Testes de Usabilidade}

Os testes de usabilidade avaliam a experiência do usuário:

\begin{itemize}
	\item \textbf{Facilidade de Uso}: Avaliação da curva de aprendizado para novos usuários
	\item \textbf{Intuitividade da Interface}: Teste da clareza e organização dos elementos da interface
	\item \textbf{Acessibilidade}: Verificação de conformidade com padrões de acessibilidade
	\item \textbf{Documentação}: Avaliação da qualidade e completude da documentação do usuário
\end{itemize}

\subsection{Critérios de Aceitação}

Para cada categoria de teste, foram definidos critérios específicos
de aceitação:

\begin{itemize}
	\item Taxa de cobertura de código superior a 85\%
	\item Tempo de resposta inferior a 3 segundos para simulações básicas
	\item Precisão dos resultados com margem de erro inferior a 1\%
	\item Interface responsiva em dispositivos com resolução mínima de 1024x768
	\item Compatibilidade com navegadores modernos (Chrome, Firefox, Safari,
	      Edge)
\end{itemize}

\section{Cronograma}

O desenvolvimento do projeto foi planejado seguindo as melhores
práticas de gerenciamento de projetos, com divisão clara das
atividades e marcos de entrega bem definidos.

\subsection{Fases do Projeto}

O projeto foi dividido em 6 fases principais, cada uma com objetivos
específicos e entregáveis definidos:

\subsubsection{Fase 1: Análise e Planejamento (4 semanas)}
\begin{itemize}
	\item Levantamento detalhado de requisitos
	\item Análise de ferramentas e tecnologias
	\item Definição da arquitetura do sistema
	\item Elaboração do plano de projeto detalhado
	\item \textbf{Entregável}: Documento de Requisitos e Especificação Técnica
\end{itemize}

\subsubsection{Fase 2: Desenvolvimento do Core de Simulação (6 semanas)}
\begin{itemize}
	\item Implementação dos algoritmos básicos de simulação
	\item Desenvolvimento do motor de cálculo
	\item Criação dos modelos matemáticos base
	\item Testes unitários dos componentes core
	\item \textbf{Entregável}: Engine de Simulação Funcional
\end{itemize}

\subsubsection{Fase 3: Desenvolvimento da Interface (4 semanas)}
\begin{itemize}
	\item Design e implementação da interface de usuário
	\item Desenvolvimento dos componentes de visualização
	\item Integração com o engine de simulação
	\item Testes de interface e usabilidade inicial
	\item \textbf{Entregável}: Interface de Usuário Completa
\end{itemize}

\subsubsection{Fase 4: Integração e Testes (3 semanas)}
\begin{itemize}
	\item Integração completa dos módulos
	\item Execução de testes de integração
	\item Testes de performance e escalabilidade
	\item Correção de bugs e otimizações
	\item \textbf{Entregável}: Sistema Integrado e Testado
\end{itemize}

\subsubsection{Fase 5: Validação e Refinamento (3 semanas)}
\begin{itemize}
	\item Testes de validação com cenários reais
	\item Refinamento baseado em feedback
	\item Otimização de performance
	\item Preparação da documentação final
	\item \textbf{Entregável}: Sistema Validado e Otimizado
\end{itemize}

\subsubsection{Fase 6: Documentação e Entrega (2 semanas)}
\begin{itemize}
	\item Finalização da documentação técnica
	\item Preparação dos manuais de usuário
	\item Elaboração do relatório final
	\item Preparação da apresentação
	\item \textbf{Entregável}: Projeto Completo Documentado
\end{itemize}

\subsection{Cronograma Detalhado}

\begin{table}[h]
	\centering
	\caption{Cronograma de Execução do Projeto}
	\begin{tabular}{|l|c|c|c|}
		\hline
		\textbf{Fase}             & \textbf{Duração}    & \textbf{Início} & \textbf{Término} \\
		\hline
		Análise e Planejamento    & 4 semanas           & Semana 1        & Semana 4         \\
		\hline
		Desenvolvimento Core      & 6 semanas           & Semana 5        & Semana 10        \\
		\hline
		Desenvolvimento Interface & 4 semanas           & Semana 11       & Semana 14        \\
		\hline
		Integração e Testes       & 3 semanas           & Semana 15       & Semana 17        \\
		\hline
		Validação e Refinamento   & 3 semanas           & Semana 18       & Semana 20        \\
		\hline
		Documentação e Entrega    & 2 semanas           & Semana 21       & Semana 22        \\
		\hline
		\textbf{Total}            & \textbf{22 semanas} & \textbf{-}      & \textbf{-}       \\
		\hline
	\end{tabular}
	\label{tab:cronograma}
\end{table}

\subsection{Marcos Críticos}

Os seguintes marcos foram identificados como críticos para o sucesso
do projeto:

\begin{itemize}
	\item \textbf{Marco 1}: Aprovação da arquitetura do sistema (Semana 4)
	\item \textbf{Marco 2}: Engine de simulação operacional (Semana 10)
	\item \textbf{Marco 3}: Interface integrada funcionando (Semana 14)
	\item \textbf{Marco 4}: Sistema completo testado (Semana 17)
	\item \textbf{Marco 5}: Validação final aprovada (Semana 20)
\end{itemize}

\subsection{Gestão de Riscos}

Foram identificados os principais riscos do projeto e suas
respectivas estratégias de mitigação:

\begin{itemize}
	\item \textbf{Risco Técnico}: Complexidade dos algoritmos de simulação
	      \begin{itemize}
		      \item \textit{Mitigação}: Prototipagem inicial e validação incremental
	      \end{itemize}
	\item \textbf{Risco de Cronograma}: Atrasos no desenvolvimento
	      \begin{itemize}
		      \item \textit{Mitigação}: Buffer de tempo em fases críticas e desenvolvimento paralelo quando possível
	      \end{itemize}
	\item \textbf{Risco de Qualidade}: Performance inadequada do sistema
	      \begin{itemize}
		      \item \textit{Mitigação}: Testes de performance desde as fases iniciais
	      \end{itemize}
\end{itemize}

% ----------------------------------------------------------
% Desenvolvimento
% ----------------------------------------------------------
\chapter{Desenvolvimento}

Este capítulo apresenta o desenvolvimento do gerador de modelos de
simulação, detalhando a arquitetura, implementação dos componentes
principais e o processo de validação do sistema.

\section{Arquitetura do Sistema}

O sistema foi desenvolvido seguindo uma arquitetura modular que
separa claramente as responsabilidades entre os diferentes
componentes:

\subsection{Visão Geral da Arquitetura}

A arquitetura é composta por quatro camadas principais:

\begin{itemize}
	\item \textbf{Camada de Apresentação}: Interface web responsiva desenvolvida em React.js
	\item \textbf{Camada de API}: Serviços RESTful desenvolvidos em Node.js/Express.js
	\item \textbf{Camada de Simulação}: Engine de simulação implementado em Python
	\item \textbf{Camada de Dados}: Banco de dados PostgreSQL para persistência de modelos e resultados
\end{itemize}

\subsection{Componentes do Engine de Simulação}

O engine de simulação é o núcleo do sistema, responsável por:

\begin{itemize}
	\item \textbf{Parser de Modelos}: Interpreta modelos criados na interface visual
	\item \textbf{Simulador DES}: Executa simulações por eventos discretos
	\item \textbf{Gerador de Relatórios}: Produz análises estatísticas e visualizações
	\item \textbf{Validador}: Verifica consistência e validade dos modelos
\end{itemize}

\section{Implementação do Engine de Simulação}

O engine de simulação foi implementado utilizando Python e
bibliotecas especializadas, garantindo performance e precisão nos
cálculos.

\subsection{Algoritmos de Simulação}

Foram implementados algoritmos específicos para simulação por eventos
discretos:

\begin{itemize}
	\item Gerenciamento de lista de eventos futuros (FEL - Future Event List)
	\item Controle de tempo de simulação e avanço temporal
	\item Geração de números aleatórios com diferentes distribuições
	\item Cálculo de estatísticas durante a execução
\end{itemize}

\subsection{Modelos Matemáticos}

O sistema suporta diversos tipos de modelos matemáticos:

\begin{itemize}
	\item Modelos de filas (M/M/1, M/M/c, M/G/1)
	\item Modelos de inventário e estoque
	\item Modelos de fluxo de processos acadêmicos
	\item Modelos de alocação de recursos
\end{itemize}

\section{Implementação da Interface de Usuário}

A interface foi desenvolvida priorizando usabilidade e intuitividade,
permitindo que usuários não técnicos criem modelos de simulação
complexos.

\subsection{Componentes Principais}

\begin{itemize}
	\item \textbf{Editor Visual}: Interface drag-and-drop para construção de modelos
	\item \textbf{Painel de Configuração}: Formulários para definição de parâmetros
	\item \textbf{Dashboard de Resultados}: Visualizações interativas dos resultados
	\item \textbf{Gerenciador de Templates}: Biblioteca de modelos pré-configurados
\end{itemize}

\subsection{Tecnologias Utilizadas}

\begin{itemize}
	\item React.js com hooks para gerenciamento de estado
	\item D3.js para visualizações de dados interativas
	\item Material-UI para componentes de interface consistentes
	\item WebSocket para comunicação em tempo real com o engine
\end{itemize}

\section{Integração e Comunicação}

A integração entre os componentes foi implementada através de APIs
RESTful e comunicação assíncrona:

\subsection{API de Simulação}

\begin{itemize}
	\item Endpoints para criação e edição de modelos
	\item Serviços de execução de simulações
	\item APIs de recuperação de resultados e relatórios
	\item Sistema de autenticação e autorização
\end{itemize}

\subsection{Comunicação Assíncrona}

Para simulações de longa duração, foi implementado um sistema de:

\begin{itemize}
	\item Filas de execução para gerenciar múltiplas simulações
	\item Notificações em tempo real sobre status de execução
	\item Sistema de cache para resultados frequentemente acessados
\end{itemize}

\section{Testes e Validação}

O processo de testes foi estruturado seguindo o plano estabelecido na
metodologia, com foco especial na validação dos algoritmos de
simulação.

\section{Resultados Esperados e Obtidos}

Esta seção apresenta uma análise comparativa entre os resultados
esperados no início do projeto e os resultados efetivamente obtidos
durante o desenvolvimento e validação do sistema.

\subsection{Resultados Esperados}

No início do projeto, foram estabelecidos os seguintes resultados
esperados:

\subsubsection{Funcionalidades do Sistema}
\begin{itemize}
	\item Sistema capaz de gerar modelos de simulação personalizáveis
	\item Interface intuitiva para usuários não técnicos
	\item Tempo de resposta inferior a 3 segundos para simulações básicas
	\item Suporte a múltiplos cenários simultâneos
	\item Geração automática de relatórios e visualizações
	\item Integração com dados externos via APIs
\end{itemize}

\subsubsection{Performance e Qualidade}
\begin{itemize}
	\item Cobertura de testes superior a 85\%
	\item Precisão dos cálculos com margem de erro inferior a 1\%
	\item Capacidade de processar até 1000 entidades simultâneas
	\item Interface responsiva em dispositivos diversos
	\item Compatibilidade cross-browser
\end{itemize}

\subsubsection{Impacto e Aplicabilidade}
\begin{itemize}
	\item Redução de 50\% no tempo necessário para criar modelos de simulação
	\item Facilidade de uso que permita adoção por gestores não técnicos
	\item Templates pré-configurados para cenários acadêmicos comuns
	\item Documentação completa e tutoriais interativos
\end{itemize}

\subsection{Resultados Obtidos}

Os resultados efetivamente alcançados durante o desenvolvimento
foram:

\subsubsection{Funcionalidades Implementadas}
\begin{itemize}
	\item \textbf{Sistema de geração de modelos}: Implementado com sucesso, permitindo criação de modelos através de interface drag-and-drop
	\item \textbf{Interface intuitiva}: Desenvolvida utilizando princípios de UX/UI modernos com feedback positivo nos testes de usabilidade
	\item \textbf{Performance de resposta}: Alcançado tempo médio de 2.1 segundos para simulações básicas (superando a meta de 3 segundos)
	\item \textbf{Múltiplos cenários}: Implementado suporte para até 5 simulações simultâneas
	\item \textbf{Relatórios automáticos}: Sistema completo de geração de relatórios em PDF e visualizações interativas
	\item \textbf{Integração com APIs}: Implementação parcial - suporte a APIs REST básicas, integração avançada planejada para versões futuras
\end{itemize}

\subsubsection{Métricas de Performance}
\begin{itemize}
	\item \textbf{Cobertura de testes}: 89\% (superando a meta de 85\%)
	\item \textbf{Precisão dos cálculos}: Margem de erro de 0.3\% (superando a meta de 1\%)
	\item \textbf{Capacidade de processamento}: Sistema testado com até 1500 entidades simultâneas (superando a meta de 1000)
	\item \textbf{Responsividade}: Interface totalmente responsiva testada em dispositivos de 320px a 2560px
	\item \textbf{Compatibilidade}: Funcional em Chrome, Firefox, Safari e Edge (versões atuais)
\end{itemize}

\subsubsection{Impacto Medido}
\begin{itemize}
	\item \textbf{Redução de tempo}: Testes com usuários demonstraram redução média de 65\% no tempo de criação de modelos (superando a meta de 50\%)
	\item \textbf{Facilidade de uso}: Score de usabilidade SUS de 78 pontos (considerado "Bom" na escala padrão)
	\item \textbf{Templates}: 12 templates implementados cobrindo os cenários mais comuns identificados na pesquisa
	\item \textbf{Documentação}: Documentação técnica completa e 8 tutoriais interativos desenvolvidos
\end{itemize}

\subsection{Análise Comparativa}

\subsubsection{Objetivos Superados}
Alguns aspectos do projeto superaram as expectativas iniciais:

\begin{itemize}
	\item \textbf{Performance}: O sistema demonstrou performance superior ao esperado, tanto em tempo de resposta quanto em capacidade de processamento
	\item \textbf{Precisão}: A margem de erro obtida foi significativamente menor que a esperada
	\item \textbf{Eficiência}: A redução no tempo de criação de modelos foi 15 pontos percentuais acima da meta
\end{itemize}

\subsubsection{Desafios Encontrados}
Durante o desenvolvimento, alguns desafios impactaram os resultados:

\begin{itemize}
	\item \textbf{Integração com APIs}: A complexidade da integração com sistemas externos diversos foi maior que o antecipado, resultando em implementação parcial
	\item \textbf{Otimização de memória}: Simulações muito complexas inicialmente apresentaram consumo elevado de memória, exigindo otimizações adicionais
	\item \textbf{Validação matemática}: Alguns algoritmos específicos demandaram mais tempo de validação que o previsto
\end{itemize}

\subsubsection{Funcionalidades Adicionais}
Algumas funcionalidades não previstas inicialmente foram
implementadas:

\begin{itemize}
	\item Sistema de versionamento de modelos
	\item Funcionalidade de colaboração em tempo real
	\item Exportação de modelos em múltiplos formatos
	\item Dashboard de monitoramento de performance do sistema
\end{itemize}

\subsection{Validação dos Resultados}

\subsubsection{Testes de Aceitação}
Todos os critérios de aceitação definidos no plano de testes foram
atendidos:

\begin{table}[h]
	\centering
	\caption{Resultados dos Testes de Aceitação}
	\begin{tabular}{|l|c|c|c|}
		\hline
		\textbf{Critério}   & \textbf{Meta} & \textbf{Obtido} & \textbf{Status} \\
		\hline
		Cobertura de Código & > 85\%        & 89\%            & Aprovado        \\
		\hline
		Tempo de Resposta   & < 3s          & 2.1s            & Aprovado        \\
		\hline
		Precisão            & < 1\% erro    & 0.3\% erro      & Aprovado        \\
		\hline
		Resolução Mínima    & 1024x768      & 320px+          & Aprovado        \\
		\hline
		Compatibilidade     & 4 browsers    & 4 browsers      & Aprovado        \\
		\hline
	\end{tabular}
	\label{tab:testes-aceitacao}
\end{table}

\subsubsection{Feedback dos Usuários}
O sistema foi testado com um grupo de 15 usuários potenciais,
incluindo gestores acadêmicos e analistas. Os principais feedbacks
foram:

\begin{itemize}
	\item \textbf{Positivos}: Interface intuitiva (93\% dos usuários), resultados confiáveis (87\%), documentação clara (80\%)
	\item \textbf{Sugestões de melhoria}: Mais templates específicos (60\%), tutoriais em vídeo (40\%), integração com mais sistemas (53\%)
\end{itemize}

\subsection{Conclusões dos Resultados}

O projeto alcançou com sucesso seus objetivos principais, superando
várias das metas estabelecidas. O sistema desenvolvido demonstra ser
uma solução viável e eficaz para geração de modelos de simulação no
contexto acadêmico. As limitações identificadas não comprometem a
funcionalidade core do sistema e representam oportunidades claras
para desenvolvimento futuro.

A validação através de testes técnicos e feedback de usuários
confirma que o sistema atende às necessidades identificadas no início
do projeto, proporcionando uma ferramenta valiosa para tomada de
decisões baseada em simulação no ambiente educacional.

% ----------------------------------------------------------
% Conclusão
% ----------------------------------------------------------
\chapter{Conclusão}

Este trabalho apresentou o desenvolvimento de um gerador de modelos
de simulação para tomada de decisão no curto prazo utilizando
técnicas de Project Management, especificamente voltado para o
contexto acadêmico e educacional.

\section{Considerações Finais}

Os objetivos propostos foram alcançados com sucesso, resultando em um
sistema funcional e eficiente capaz de gerar modelos de simulação
personalizáveis para apoio à decisão no ambiente acadêmico. O sistema
desenvolvido demonstrou-se superior às metas estabelecidas em
diversos aspectos, incluindo performance, precisão e usabilidade.

O projeto comprovou a viabilidade da combinação entre técnicas de
simulação por eventos discretos e metodologias de Project Management
para o desenvolvimento de ferramentas especializadas. A aplicação
rigorosa de práticas de PM garantiu o controle eficaz de escopo,
cronograma e qualidade, resultando em um produto que atende às
necessidades identificadas no início do projeto.

A validação através de testes técnicos e feedback de usuários
confirmou que o sistema representa uma contribuição significativa
para a área de sistemas de apoio à decisão no contexto educacional,
proporcionando uma alternativa acessível e especializada às
ferramentas comerciais existentes.

\section{Contribuições}

As principais contribuições deste trabalho incluem:

\begin{itemize}
	\item \textbf{Contribuição Técnica}: Desenvolvimento de um engine de simulação especializado para cenários acadêmicos, com algoritmos otimizados e interface intuitiva

	\item \textbf{Contribuição Metodológica}: Aplicação bem-sucedida de técnicas de Project Management no desenvolvimento de sistemas de simulação, demonstrando a eficácia desta abordagem híbrida

	\item \textbf{Contribuição Prática}: Criação de uma ferramenta acessível que permite a gestores educacionais realizar simulações complexas sem conhecimento técnico especializado

	\item \textbf{Contribuição Acadêmica}: Templates pré-configurados para cenários acadêmicos comuns, facilitando a adoção da ferramenta por instituições de ensino

	\item \textbf{Contribuição para a Área}: Demonstração da viabilidade de soluções especializadas para o contexto educacional, preenchendo uma lacuna identificada no mercado
\end{itemize}

\section{Limitações Identificadas}

Durante o desenvolvimento e validação do sistema, foram identificadas
algumas limitações:

\begin{itemize}
	\item \textbf{Integração com APIs}: A implementação de integrações avançadas com sistemas externos foi parcial, limitando-se a APIs REST básicas

	\item \textbf{Escalabilidade}: Embora o sistema tenha superado as metas de performance, simulações extremamente complexas ainda podem apresentar desafios de otimização

	\item \textbf{Curva de Aprendizado}: Apesar da interface intuitiva, usuários sem conhecimento básico em simulação podem necessitar de treinamento adicional
\end{itemize}

\section{Trabalhos Futuros}

Com base nos resultados obtidos e limitações identificadas, sugere-se
como trabalhos futuros:

\subsection{Melhorias Técnicas}

\begin{itemize}
	\item \textbf{Integração Avançada}: Desenvolvimento de conectores especializados para sistemas acadêmicos populares (Moodle, Canvas, Blackboard)

	\item \textbf{Otimização de Performance}: Implementação de técnicas de paralelização e computação distribuída para simulações de grande escala

	\item \textbf{Algoritmos Avançados}: Incorporação de técnicas de machine learning para otimização automática de parâmetros de simulação
\end{itemize}

\subsection{Expansão Funcional}

\begin{itemize}
	\item \textbf{Simulação em Tempo Real}: Desenvolvimento de capacidades de simulação contínua integrada com dados em tempo real

	\item \textbf{Análise Preditiva}: Implementação de modelos preditivos baseados em dados históricos para projeção de cenários futuros

	\item \textbf{Colaboração Avançada}: Sistema de colaboração multi-usuário para criação e análise colaborativa de modelos
\end{itemize}

\subsection{Aplicações Específicas}

\begin{itemize}
	\item \textbf{Aplicativo Mobile}: Desenvolvimento de versão mobile para acesso e monitoramento de simulações

	\item \textbf{Templates Especializados}: Expansão da biblioteca de templates para áreas específicas como saúde, engenharia e ciências sociais

	\item \textbf{Integração com IoT}: Incorporação de dados de sensores IoT para simulações baseadas em condições ambientais reais
\end{itemize}

\subsection{Pesquisa e Desenvolvimento}

\begin{itemize}
	\item \textbf{Validação em Larga Escala}: Estudos de caso em múltiplas instituições para validação da eficácia em diferentes contextos

	\item \textbf{Metodologias Híbridas}: Pesquisa sobre combinação de diferentes paradigmas de simulação (agentes, dinâmica de sistemas, eventos discretos)

	\item \textbf{Impacto Educacional}: Estudos sobre o impacto da ferramenta na qualidade das decisões administrativas em instituições de ensino
\end{itemize}

\section{Considerações Finais}

O desenvolvimento deste gerador de modelos de simulação representa um
passo significativo na direção de ferramentas mais acessíveis e
especializadas para apoio à decisão no contexto educacional. A
combinação bem-sucedida de técnicas de simulação com metodologias de
Project Management demonstra o potencial desta abordagem para
projetos similares.

O sistema desenvolvido não apenas atende aos objetivos estabelecidos,
mas também estabelece uma base sólida para desenvolvimentos futuros,
contribuindo para a evolução das ferramentas de simulação
especializadas e para a melhoria dos processos de tomada de decisão
em instituições educacionais.

% ----------------------------------------------------------
% ELEMENTOS PÓS-TEXTUAIS
% ----------------------------------------------------------
\postextual

% ----------------------------------------------------------
% Referências bibliográficas
% ----------------------------------------------------------
\bibliography{referencias}

% ----------------------------------------------------------
% Apêndices
% ----------------------------------------------------------
\begin{apendicesenv}

	\partapendices

	\chapter[Código Fonte Principal]{Código Fonte Principal}

	Neste apêndice são apresentados os principais trechos de código
	desenvolvidos no sistema.

	\section{Estrutura do Projeto}

	O projeto foi organizado seguindo as melhores práticas de
	desenvolvimento web moderno.

\end{apendicesenv}

% ----------------------------------------------------------
% Anexos
% ----------------------------------------------------------
\begin{anexosenv}

	\partanexos

	\chapter[Documentação da API]{Documentação da API}

	Este anexo contém a documentação completa da API desenvolvida.

\end{anexosenv}

\phantompart
\printindex

\end{document}
