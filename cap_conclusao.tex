\chapter{Conclusão}

Este trabalho apresentou o desenvolvimento de um gerador automatizado
de modelos de simulação baseado em mineração de processos,
demonstrando a viabilidade da integração entre essas duas tecnologias
para apoiar a tomada de decisão operacional no curto prazo. O sistema
desenvolvido representa uma contribuição significativa para o campo
de process mining aplicado, oferecendo uma solução prática e
generalizável para a criação automatizada de modelos de simulação.

\section{Limitações e Desafios}

Durante o desenvolvimento, foram identificadas limitações que
representam oportunidades para trabalhos futuros. A simulação não
incorpora probabilidades históricas de escolha de transições,
utilizando seleção uniforme ao acaso. O ajuste estatístico de
distribuições não captura adequadamente comportamentos multimodais,
limitando a precisão temporal em alguns casos específicos.

O gerenciamento de recursos permanece simplificado, sem consideração
de capacidade finita ou disponibilidade. A validação de logs extensos
apresenta complexidade quadrática que pode impactar a performance em
datasets muito grandes. Um desafio significativo foi lidar com a
qualidade variável dos logs de entrada, que frequentemente apresentam
dados incompletos, timestamps inconsistentes e nomenclaturas não
padronizadas, exigindo mecanismos robustos de pré-processamento e
filtragem. Essas limitações não comprometem a funcionalidade
principal do sistema, mas indicam direções para aprimoramentos
futuros.

\section{Considerações Finais}

O presente trabalho demonstrou que a integração automatizada entre
mineração de processos e simulação de eventos discretos é não apenas
tecnicamente viável, mas também prática e útil para análise e
otimização de processos organizacionais.

Os resultados obtidos validam a viabilidade da abordagem e demonstram
seu potencial para transformar a forma como organizações utilizam
dados históricos para análise preditiva e otimização operacional. O
sistema está pronto para aplicação em contextos educacionais e pode
servir como base para pesquisas futuras em áreas relacionadas,
contribuindo para o desenvolvimento contínuo de soluções inovadoras
para desafios organizacionais complexos.
% ----------------------------------------------------------
% ELEMENTOS PÓS-TEXTUAIS
% ----------------------------------------------------------
